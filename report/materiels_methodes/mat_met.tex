\section{Présentation générale de la Zone d'étude}

\section{Données satellitaires utilisées}

Notre série temporelle est composée de 3 types d’imageries satellitaires ont été utilisées dans ce travail. Il s'agit d'images PlanetScope, RapidEye et Sentinel-2. 

  \subsection{Imagerie PlanetScope et RapidEye}
  
Les imageries PlanetScope sont produites par la société privée américaine Planet Labs, Inc. \footnote{\url{https://www.planet.com/}} fondée en 2010, basée à San Francisco 
en Californie et spécialisée dans l'observation de la terre par imagerie satellitaire. Planet Labs conçoit et fabrique des \emph{nanosatellites} ou \emph{CubSats} qui sont placés sur 
orbite en tant que charge utile secondaire sur d'autres missions de lancement de fusée. La société dispose ainsi d'une constellation de nanosatellites appelés \emph{Doves} notamment 
\emph{Flock}. La constellation est composée de 88 nanosatellites au total en 2017. Ces nanosatellites produisent des images complètes de la Terre une fois par jour à une résolution 
spatiale de 3 à 5 mètres. Ces images fournissent des informations permettant de suivre les changements climatiques, de prévoir les récoltes, de gérer les catastrophes ou encore de 
mettre au point des applications urbaines. Les images recueillies par les Doves sont accessibles en ligne et certaines disponibles dans le cadre de l'Open Data. \\
En 2015, Planet Labs a acquis la constellation RapidEye auprès société allemande BlackBridge. Les images RapidEye qui ont une résolution spatiale de 5 mètres sont fournies par une 
constellation de 5 satellites. Leur période de revisite est de 5,5 jours au nadir ou journalière sinon. Enfin en 2017, Google vend sa filiale Terra Bella et sa constellation de 
satellites \emph{SkySat} à Planet Labs. Les images SkySat ont une résolution submétrique (80 cm). 

    \paragraph{Spécifications des images PlanetScope}

Trois niveaux de traitements sont disponibles pour les images PlanetScope. Ce sont les niveaux 1B, 3A et 3B. Le niveau 1B correspond aux produits basiques. Les données numériques ont 
été calibrées en radiance \acrshort{toa} mais les images ne sont pas géoréférencées. Le niveau 3B qui est celui de nos images correspond à des produits orthorectifiés et projetés en
\acrshort{utm}. Comme pour le niveau 1B, les valeurs numériques ont été calibrées en radiance TOA. Le niveau 3A est similaire au 3B à la différence que les images sont tuilées pour 
couvrir un système de grilles de $25\times25$ kilomètres.\\
Les images PlanetScope sont distribuées en format \emph{GeoTiff}. Au niveau 3B, elles ont une résolution au sol de \emph{3 mètres} et une résolution radiométrique de \emph{12 bits} s'il 
s'agit de compte numérique ou \emph{16 bits} dans le cas des radiances TOA. Elles disposent de 4 bandes spectrales (\Cref{planetscope}). 

\begin{table}
\begin{center}
\caption{Caractéristiques spectrales des images PlanetScope}
\label{planetscope}
 \begin{tabular}{ccc}
  \hline
  Bande spectrale & Domaine spectral & Longueurs d'onde (micromètres)\\
  \hline
  1 & Bleu & 0,455 --- 0,515 \\
  2 & Vert & 0,500 --- 0,590 \\
  3 & Rouge & 0,590 --- 0,670 \\
  4 & Proche Infrarouge & 0,780 --- 0,860 \\ 
  \hline
 \end{tabular}
\end{center}
\end{table}

La conversion des valeurs de pixels en réflectance TOA pour les images PlanetScope est donnée par la formule suivante :
\[
   Reflectance (i) = DN(i) \times reflectanceFactor(i)
\]

  \paragraph{Spécifications des images RapidEye}

Les images RapidEye sont pour leur part, disponibles en 2 niveaux de traitements : 1B et 3A. Ces niveaux de traitements sont identiques à ceux des produits PlanetScope. Les images
RapidEye utilisées sont traitées au niveau 3A. Elles sont distribuées également en format \emph{GeoTiff}. Leur résolution au sol est de \emph{5 mètres} et leur résolution 
radiométrique de \emph{16 bits}. Elles disposent de 5 bandes spectrales (\Cref{rapideye}).

\begin{table}
\begin{center}
\caption{Caractéristiques spectrales des images RapidEye}
\label{rapideye}
 \begin{tabular}{ccc}
  \hline
  Bande spectrale & Domaine spectral & Longueurs d'onde (micromètres)\\
  \hline
  1 & Bleu & 0,440 --- 0,510 \\
  2 & Vert & 0,520 --- 0,590 \\
  3 & Rouge & 0,630 --- 0,685 \\
  4 & Red Edge & 0,690 --- 0,730 \\ 
  5 & Proche Infrarouge & 0,760 --- 0,850 \\
  \hline
 \end{tabular}
\end{center}
\end{table}

La conversion des valeurs de pixels en réflectance TOA pour les produits RapidEye se fait en 2 étapes :
\[  Radiance (i) = DN(i) \times radiometricScaleFactor(i) \]

\[   Reflectance(i) = Radiance(i) \times \pi \]

  \subsection{Imagerie Sentinel-2}
 
\vspace{5mm}
 
Nous avons mis en place une chaîne de traitements écrit en \texttt{Python} découpage mosaiquage
  
\section{Données de terrain}
  
\section{Extraction d'indices spectraux}

\section{Lissage et Interpolation de la Série temporelle}

\section{Extraction des Métriques Phénologiques}
afin d'évaluer les dates de semis
Compte tenu de notre objectif qui est celui d'atteindre un suivi spatialisé des systèmes de cultures,

\section{Estimation des Rendements}