\section{Extraction des dates de semis par Télédétection}
  
  \subsection{Considérations générales}
  
Les périodes ou dates les plus appropriés pour mettre en terre des semences dépendent de nombreux facteurs.  Nous pouvons considérer les espèces et leurs variétés, les températures 
selon la zone de prodcution, l'humidité du sol, les objectifs de la production (date de récolte souhaitée) ou encore les méthodes de production (plein air, serre). Cependant, l'on peut
s'accorder sur le rôle prépondérant que joue le facteur climatique. Au Sénégal comme dans d'autres pays d'Afrique où l'agriculture est pluviale, le comportement des agriculteurs est
généralement celui de semer après les premiers épisodes pluvieux importants. Ceux-ci peuvent être néanmoins emmenés à semer de nouveau quand survient un épisode sec risquant de
mettre en péril les premières semences.\\ \'Etant donné l'étroite relation entre les rendements agricoles et la longueur de la saison, des semis tardifs sont susceptibles de 
conduire à des rendements faibles en fin de saison. Le suivi du démarrage de la saison agricole permet donc de fournir aux décideurs, une évaluation précoce des menaces potentielles 
à la production et à la sécurité alimentaire. La méthode la plus courante pour l'estimation des dates de semis dans les pays d'Afrique de l'Ouest est un approche agrométéorologique 
qui consiste à appliquer des seuils sur les quantités de précipitations survenues pendant une période définie \citep{Marinho2014}. Cependant, cette méthode présente des inconvénients liés 
à la résolution spatiale des données qui est souvent grossière (de l'ordre des kilomètres) et à l'estimation en elle même des quantités de précipitations qui peut être imprécise. 
Compte tenu de notre échelle de travail qui est parcellaire, cette méthode n'est pas adaptée à notre étude et ne sera pas considérée. \\Une autre approche par Télédection cette fois-ci
consiste à dériver les dates du démarrage de la végétation ou en général les métriques phénologiques à partir d'indices de végétation comme le \acrshort{ndvi} ou le \acrshort{evi} 
et à les lier aux dates de semis. Ces indices de végétation rendent compte entre autres de l’activité photosynthétique de la
végétation, de l’intensité de son métabolisme ainsi que de sa verdure \citep{Duarte2018}. Cette approche par Télédétection présente l'avantage d'avoir non seulement une résolution spatiale 
beaucoup plus élevée en fonction de l'image 
satellitaire utilisée mais intègre également la réponse spectrale de la végétation aux divers facteurs externes, pratiques agricoles y compris. Bien évidemment, les contraintes intrinsèques à 
l'acquisition des images satellitaires ne sont pas en reste. C'est notamment le cas des bruits induits par la diffusion atmosphérique. Il faudra donc avant exploitation des images, considérer 
par exemple l'application d'une technique de lissage. Une autre précaution est à prendre vis-à-vis de l'influence des sols nus dans la réponse spectrale du couvert végétal 
notamment quand celui ci est clairsemé. Il faudra considérer par exemple l'utilisation d'un indice de végétation tenant compte de l'effet des sols. 
Il existe une foultitude de méthodes dans la littérature pour estimer le démarrage de la végétation en étudiant sa phénologie par observation satellitaire. 
  
  \subsection{Phénologie de la végétation}
  
La \emph{phénologie} 
est l’étude de l’apparition d’événements périodiques (annuels le plus souvent) dans le monde vivant, déterminée par les variations saisonnières du climat. Elle se traduit au niveau de 
la végétation, par l’ensemble des stades de développement intervenant dans le cycle de vie des plantes en l’occurrence le bourgeonnement, la croissance, la floraison et la 
sénescence \citep{Kimball2014}. La phénologie du Mil peut être par exemple décomposée en une phase \emph{végétative}, une phase \emph{reproductrice} et une phase de 
\emph{maturation}. Selon les variétés de Mil cultivées, la longueur du cycle de développement peut être de 90 à 100 jours pour les \emph{Souna} (Variétés à cycle court --- 
petits grains, épis non aristés et insensibles à la longueur du jour) ou de 130 à 150 jours pour les \emph{Sanio} (Variétés à cycle long --- grains plus gros, épis aristés et sensibles
à une photopériode de jours courts) \citep{Diouf2001}.
\\L'\'etude de la phénologie des plantes sur une échelle régionale à globale à partir d’observations satellitaires est désignée par \acrfull{lsp} \citep{Helman2018}. Le terme LSP 
vient tenir compte du fait que le signal intercepté par les satellites provient d’une surface hétérogène et n’est pas représentatif de la réponse spectrale d’une seule espèce 
\citep{Kimball2014}. L'analyse de la LSP contribue entre autres à diverses applications comme l’étude des changements climatiques \citep{Begue2014} ou encore le suivi des
cultures. Les séries temporelles d’indices de végétation provenant de capteurs comme AVHRR, MODIS, SPOT VEGETATION ou encore PROBA-V permettent donc d’évaluer la phénologie de la 
végétation sur une échelle régionale à globale. Plus récemment, quelques auteurs ont étudié la phénologie de la végétation à une échelle beaucoup plus fine en utilisant des données
provenant de satellites comme HJ-1 A/B \citep{Pan2015} ou RapidEye \citep{Vrieling2017}.

  \subsection{Métriques Phénologiques}

 

\section{Estimation des rendements par Télédétection}