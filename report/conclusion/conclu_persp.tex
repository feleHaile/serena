Ce travail a été réalisé dans le contexte de l’évaluation spatialisée par télédétection
des systèmes de cultures à base de mil et d’arachide dans le bassin arachidier du
Sénégal. Dans cette perspective, il avait pour objectif principal d’évaluer le potentiel
d’une série temporelle multisource (Sentinel-2, RapidEye et PlanetScope) à travers
deux sous objectifs : l’évaluation des dates de semis et l’estimation des biomasses vé-
gétatives et des rendements grains du mil et gousses de l’arachide.
\\En lissant les profils temporels de NDVI extraits à partir de la série multisource,
notamment avec la méthode de Whittaker, nous avons extrait différentes métriques
phénologiques dont les dates de début de croissance de la végétation (SOS). Avec les
SOS extraits, nous avons estimé les dates de semis des différents systèmes de culture
avec notamment plus ou moins 5 jours de décalage pour les parcelles d’arachide et
10 à 20 jours de décalage pour celles de mil. Pour l’estimation des biomasses et ren-
dements, le NDVI, qui est traditionnellement utilisé, a été comparé à un autre indica-
teur de productivité de la végétation, le GDVI. Nous avons montré que le NDVI était
moins performant que le GDVI pour expliquer la variabilité de la biomasse et des
rendements, ceci s’expliquant notamment par un changement dans les technologies
d’observation de la Terre par rapport aux études précédentes. Cette étude a montré
le réel intérêt de combiner différentes sources d’images satellitaires afin d’améliorer
la répétitivité de l’information spatiale. Elle a également montré que la différence de
résolutions spatiales entre les types d’images ne représente en aucun cas un frein à
leur utilisation combinée au sein d’une même application.
\\Néanmoins, au vu des résultats obtenus, la méthodologie mise en place pourrait être
affinée, notamment pour améliorer la partie estimation des biomasses et rendements.
Dans un ordre de priorité, notre première recommandation concerne la phase de
prétraitements des images qui pourrait inclure des corrections atmosphériques afin
d’uniformiser tout le jeu données. En effet, seules les images Sentinel-2A dans notre
série temporelle multisource présentaient des corrections atmosphériques et nom-
breuses sont les études en télédétection ayant fait état du gain significatif que ces
corrections pouvaient apporter. Dans notre série temporelle, nous n’avons utilisé que
les images Sentinel-2A. Sentinel-2B étant aujourd’hui opérationnel, un couplage des
2 types d’images viendrait densifier notre série temporelle afin d’éviter des périodes
sans informations comme cela a été le cas pendant le mois de Septembre dans notre
étude. Également sur ce point, l’utilisation d’images radar pourrait être une solution
envisageable quand on sait que le principal avantage de l’imagerie radar se trouve
dans l’affranchissement des conditions atmosphériques, nuages notamment. L’adop-
tion d’images radars permettrait également de tester un apport éventuel des indices
radar pour l’extraction des SOS et EOS et l’estimation des biomasses et rendements.
En ce qui concerne les méthodes de lissage, s’il est vrai qu’elles ont plutôt donné
satisfaction, il n’en est pas moins que cette étape influence fortement les résultats
obtenus par la suite. Il serait alors intéressant de tester d’autres types d’approches
comme les fonctions asymétriques gaussiennes ou les doubles fonctions logistiques dont les résultats sont jugés tout autant satisfaisants. Aussi, la grande majorité de
notre travail a été basé sur l’utilisation du NDVI qui a montré ses limites pour l’es-
timation des biomasses et rendements. Dès lors, il serait pertinent de se tourner vers
d’autres indices de végétation comme cela a déjà pu être le cas avec le GDVI. D’autres
indicateurs comme le MSAVI qui tient compte de l’effet des sols (intéressant dans
le cas de cultures peu couvrantes comme le mil), le NDWI pouvant détecter la pré-
sence de stress hydrique chez les cultures ou encore les indices explorant la bande
du red edge comme le Red Edge NDVI pourraient être de réelles alternatives. Par
ailleurs, l’extraction des métriques phénologiques a été réalisée uniquement à partir
de la méthode de seuillage relatif sur les amplitudes de NDVI et il serait intéressant
de tester d’autres méthodes notamment les méthodes dérivatives. Outre cette étape,
celle de l’estimation des biomasses et rendements a tenu compte de variables expli-
catives provenant des seuls indices que sont le NDVI et le GDVI. Nous n’avons ni
exploré les variables biophysiques comme le LAI ni les variables climatiques comme
l’humidité du sol et les quantités de précipitations bien connues pour améliorer les
modèles calibrés. Les indices texturaux pourraient être également interessant à tester.
L’autre phase inexplorée également dans notre travail concerne la non prise en compte
du pouvoir explicatif des arbres (présents dans les parcelles) dans la productivité des
systèmes de cultures. Une des limites de notre travail aura été l’étroitresse de la taille
de notre jeu de données terrain rendant impossible la calibration de modèles multi-
variés plus robustes et leur validation. En effet, une taille d’échantillons plus grande
aurait permis l’implication de plusieurs variables dans nos modèles et aurait certaine-
ment amélioré les estimations de biomasses et rendements. D’autre part, augmenter
la taille de nos échantillons permettrait d’envisager le recours à des méthodes de ré-
gression plus poussées par fouilles de données telle que les forêts aléatoires d’arbres
décisionnels pour l’estimation des biomasses et rendements. Une autre limite de notre
travail repose sur la qualité des données de biomasses et de rendements observés. Ef-
fectivement, ces données ayant été collectées dans la cadre d’autres projets et donc
pas initialement pour l’estimation des biomasses et rendements, il nous aura été im-
possible de savoir jusqu’à quel point nous y fier. Sur ce point, un nouveau réseau de
parcelles est en cours de suivi pour la campagne agricole 2018. Ce nouveau jeu de
données sera ajouté à celui de la campagne 2017 sur laquelle nous avons travaillé et
viendra élargir la taille d’échantillons pour les estimations de biomasses et de rende-
ments de 2018. Pour finir, une évaluation spatialisée des rendements dans la région
du bassin arachidier sénégalais requiera avant tout un masque des principaux sys-
tèmes de cultures présents. Ce type de données est obtenu par classification. Dans le
cadre du projet S2-Agri\footnote{\url{http://www.esa-sen2agri.org/}} , un outil a été mis en place pour la production de masques
de cultures ainsi que des cartes d’occupation du sol spécialisées sur l’agriculture.
