\section{Résultats}

\subsection{Extraction du SOS et du EOS}

\`A partir des profils temporels de NDVI obtenus par les 2 méthodes de lissage (HANTS et Whittaker), nous avons extrait le SOS et le EOS respectivement avec les seuils de 10, 20, 30 puis 50\% avant le MAX et 50, 60, 70 puis 80\% après le MAX. Nous avons calculé ensuite pour chacune des parcelles terrain, les écarts en nombre de jours entre les dates de semis et les SOS extraits puis entre les dates de récolte et les EOS estimés. En considérant ces écarts par sytème de culture, nous avons calculé 2 indicateurs statistiques : la racine carrée de l'erreur quadratique moyenne ou \acrshort{rmse} qui est un indicateur d'écart entre valeurs observées et valeurs prédites et le coefficient de variation ou \acrshort{cv} qui mesure la dispersion relative autour la moyenne. 

\paragraph{SOS et évaluation des dates de semis}
La distribution des écarts entre les dates de semis observées et les SOS extraits par système de culture et méthode de lissage est illustrée sur la \cref{fig-sosboxplot}. Globalement, les parcelles d'arachide mixte montrent les variabilités les plus faibles entre les écarts calculés (moins de 15 jours d'écart au maximum, seuils et méthodes de lissage confondus) suivies des parcelles de mil pur (moins de 20 jours d'écart au maximum, seuils et méthodes de lissage confondus) et des parcelles de mil mixte qui présentent les variabilités les plus fortes (près de 60 jours d'écart avec les SOS estimés par HANTS pour un seuil de 10\%). En analysant la distribution des écarts par méthode de lissage, nous remarquons que pour l'ensemble des systèmes de culture et presque pour tous les seuils, la plage des écarts obtenus avec les SOS estimés par HANTS est toujours plus importante que celle des écarts obtenus avec les SOS extraits par la méthode de Whittaker avec des écarts plus grands pour HANTS quand on considère le même seuil. Ceci semble indiquer que la méthode de lissage de Whittaker soit plus appropriée pour estimer le timing du démarrage de croissance de la végétation. L'analyse de la distribution des écarts en fonction du seuil d'extraction du SOS montre quant à elle une certaine tendance à la réduction de la variabilité des écarts quand le seuil d'extraction s'accroit, méthodes de lissage et systèmes de culture confondus à l'exception du mil pur. Cependant, cette tendance est à relativiser au vu de l'apparition de valeurs aberrantes à mesure que le seuil d'extraction du SOS croît et d'autant plus que les écarts entre dates de semis et SOS estimés s'accroissent également, ce qui est tout à fait logique puisque l'amplitude du NDVI avant le MAX est proportionnelle au temps. En effet, plus ces écarts seront importants et plus les SOS déterminés ne seront plus réalistes vis-à-vis du temps de germination des semences, l'inverse étant également vraie. 
Ainsi, nous pouvons remarquer que le seuil de 10\% extrait parfois des SOS précoces (écarts 
négatifs pour certaines parcelles d'arachide et de mil mixtes) et que le seuil de 50\% les extrait tardivement (40 à 60 jours après les dates de semis). Le seuil le plus adapté doit extraire les SOS avec des écarts réalistes par rapport aux dates de semis et minimiser au mieux la variabilité entre ces écarts. Afin de déterminer les seuils les plus adaptés et ce par système de culture, référons nous à la \cref{fig-sos-rmse-cv} où nous avons représenté le RMSE entre les dates de semis et SOS estimés en fonction du coefficient de variation (CV) de leurs écarts par système de culture et méthode de lissage. La variation du RMSE en fonction du seuil d'extraction  quelque soit le système de culture ou la méthode de lissage pris en compte rejoint l'analyse sur la distribution des écarts. En effet, plus le seuil d'extraction croit et plus le RMSE entre SOS et dates de semis est grand, ce dernier étant un indicateur d'écart. La variation du CV en fonction du seuil d'extraction révèle d'abord 2 valeurs qui portent à interrogation : -233\% (pour un seuil de 10\% avec la méthode de Whittaker sur les parcelles d'arachide mixte) et 342\% (pour un seuil de 10\% avec HANTS sur les parcelles de mil mixte). Notre investigation a révélé qu'il s'agissait de cas où l'écart-type était supérieure à une moyenne très petite  proche de $0$ (négative et positive respectivement). Ceci indique que les SOS extraits dans ces cas sont précoces. Pour le reste, la valeur du coefficient de variation diminue quand le seuil d'extraction croît, système de culture et méthode de lissage confondus. Ceci peut s'expliquer par le fait que la moyenne des écarts entre SOS et dates de semis augmente à mesure que le seuil d'extraction s'accroit quand l'écart-type ou la variabilité entre les écarts est sensiblement la même. Finalement, il apparait que le seuil le plus adapté pour les parcelles d'arachide mixte est celui de 20\% avec la méthode de Whittaker qui estime les SOS avec un RMSE légèrement supérieur à 10 jours. Pour les parcelles de mil pur et de mil mixte, le seuil de 10\% avec la méthode de Whittaker apparait être le plus adapté avec des RMSE de 15 jours sur les estimations de SOS. Nous avons également spatialisé les SOS extraits (\Cref{fig-spatial-sos}).
\begin{figure}[htbp]
 \begin{center}
  \includegraphics[scale=0.7]{resultats_discussions/SOS_Boxplot.png} 
 \end{center}
 \caption[Distribution des écarts entre SOS et dates de semis]{Boîtes à moustaches illustrant la distribution des écarts entre SOS estimés et dates de semis observées en fonction des systèmes de culture et des méthodes de lissage (\emph{Le nombre de parcelles est indiqué à la suite du système de culture})}
 \label{fig-sosboxplot}
\end{figure}

\begin{figure}[htbp]
 \begin{center}
  \includegraphics[scale=0.7]{resultats_discussions/SOS_RMSE_vs_CV.png} 
 \end{center}
 \caption[SOS -- RMSE vs CV]{Représentation du RMSE entre dates de semis et SOS estimés en fonction du coefficient de variation de leurs écarts par système de culture et méthode de lissage}
 \label{fig-sos-rmse-cv}
\end{figure}

\begin{figure}[htbp]
 \begin{center}
  \includegraphics[scale=0.7]{resultats_discussions/Spatial_SOS.png} 
 \end{center}
 \caption[Spatialisation des SOS]{Spatialisation des SOS extraits à l'échelle pixellaire (3 mètres) sur la zone d'application du lissage}
 \label{fig-spatial-sos}
\end{figure}

\paragraph{EOS}
Comme pour les SOS, nous avons représenté la distribution des écarts mais cette fois ci entre les dates de récoltes observées et les EOS extraits par système de culture et méthode de lissage (\Cref{fig-eosboxplot}). Les parcelles d'arachide mixte montrent légèrement moins de variabilité entre les écarts que les parcelles de mil mixte et de mil pur. La plage des écarts obtenus avec les EOS estimés par HANTS semble légèrement inférieure à celle des écarts obtenus avec les EOS extraits par la méthode de Whittaker notamment pour tous les seuils sur les parcelles d'arachide mixte. Néanmoins, les écarts obtenus avec HANTS sont nettement inférieurs à ceux de la méthode de Whittaker quand on considère le même seuil et tous les sytèmes de culture confondus. En ce qui concerne la variation des seuils, plus ils sont grands et plus les dates de fin de saison estimées se rapprochent 
des dates de récoltes que ce soit pour les parcelles d'archide ou de mil. Néanmoins, le seuil de 60\% extrait des EOS précoces pour les parcelles d'arachide mixte (médiane des écarts autour de $0$ pour les 2 méthodes de lissage) tandis que les EOS extraits pour les parcelles de mil avec un seuil de 80\% ont au moins 10 jours d'écart avec les dates de récoltes. Le seuil le plus adapté pour chaque système de culture doit déterminer les EOS en minimisant la variabilité des écarts par rapport aux dates de récoltes et en les minimisant au mieux puisqu'en fin de compte les périodes de forte corrélation pour l'estimation des rendements qui viendra par la suite, ne devraient pas excéder les dates de récoltes. Référons nous alors à la \cref{fig-eos-rmse-cv}. Pour les parcelles d'arachide mixte, les RMSE les plus faibles sont obtenus avec les seuils de 60 et 70\% ce qui est normal car les écarts entre EOS et dates de récoltes à ces seuils sont faibles. Par contre, leurs coefficients de variations très élevés rappellent le cas de dates précoces comme pour le SOS. 

\begin{figure}[htbp]
 \begin{center}
  \includegraphics[scale=0.7]{resultats_discussions/EOS_Boxplot.png} 
 \end{center}
 \caption[Distribution des écarts entre EOS et dates de récoltes]{Boîtes à moustaches illustrant la distribution des écarts entre EOS estimés et dates de récoltes observées en fonction des systèmes de culture et des méthodes de lissage}
 \label{fig-eosboxplot}
\end{figure}

\begin{figure}[htbp]
 \begin{center}
  \includegraphics[scale=0.7]{resultats_discussions/EOS_RMSE_vs_CV.png} 
 \end{center}
 \caption[EOS -- RMSE vs CV]{Représentation du RMSE entre EOS estimés et dates de récoltes observées en fonction du coefficient de variation de leurs écarts par système de culture et méthode de lissage}
 \label{fig-sos-rmse-cv}
\end{figure}


\subsection{Estimation des rendements}
  
\section{Discussions}

\subsection{Sur l'extraction du SOS et du EOS}

\subsection{Sur l'estimation des rendements}
