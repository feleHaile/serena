\section{Contexte du stage}

Le présent stage s'inscrit dans le cadre de recherches portant sur l’évaluation spatialisée des pratiques d’intensification écologique des systèmes de cultures à base de Mil au 
Sénégal, dans la région du bassin arachidier par télédétection. Ces travaux sont relatifs aux projets GloFoodS SERENA, SIMCo et TOSCA LYSA et sont conduits par 
l'\acrshort{aida} (CIRAD --- Montpellier, France) conjointement avec ses partenaires sénégalais le \acrshort{cse} et l'\acrshort{isra} (Dakar, Sénégal).

  \subsection{Présentation des organismes d'accueil}
Ce stage s'est déroulé dans les locaux du CSE.
    
    \paragraph{L'UR AÏDA}
    
    \paragraph{Le CSE}
    
    \paragraph{L'ISRA}
    
  
  \subsection{Projets GloFoodS SERENA, SIMCo et LYSA}

\section{Objectifs et Hypothèses}

L'objectif de ce stage est de mener une première analyse des potentialités offertes par une série temporelle multisource à haute résolution spatiale, pour un suivi spatialisé des 
systèmes de cultures à base de Mil, à Niakhar dans le bassin arachidier sénégalais. De manière plus concrète, il s'agira :
  \begin{itemize}
   \item d'évaluer et cartographier les dates de semis sur les différentes parcelles
   \item et de calibrer un modèle statistique d'estimation des rendements.
  \end{itemize}
 
\vspace{5mm} %5mm vertical space

Pour ce faire, nous travaillerons avec une série temporelle multisource composée d'images \emph{PlanetScope}, \emph{RapidEye} et \emph{Sentinel-2} et couvrant presque entièrement 
la saison agricole 2017. Nous disposons également de données de terrain, décrivant sur une trentaine de parcelles de la zone, les pratiques agricoles adoptées ou encore les 
caractéristiques agronomiques dont les rendements observés. La principale contrainte pour ce travail que l'on peut considérer comme anodine malgré tout, se trouve dans l'utilisation 
de solutions libres et open-source. Ce travail vient en amont d'une étude plus complète visant à évaluer à l’échelle d’un paysage, l’impact de la biodiversité paysagère sur la productivité des 
systèmes de cultures et les potentiels d’intensification des pratiques.

\vspace{5mm} %5mm vertical space

Sur base des observations dans la zone de Niakhar, nos hypothèses pour ce travail sont les suivantes :
  \begin{itemize}
   \item \'Etant donné l'apparition des premiers germes de Mil vers le mois de Juillet pour une fin de saison au 
   plus tard en fin Septembre ou début Octobre, les dates de semi devraient se situer pendant le mois de Juin. 
   \item Pour le Mil, les rendements seraient améliorés par une association culturale en l'occurrence Mil --- Niébé mais aussi par la présence d'arbres dans la parcelle.
  \end{itemize}

\section{Planning de travail}
Afin d'assurer sa bonne marche, le présent travail a été subdivisé en une série de 5 tâches principales à réaliser sur la période des 6 mois de stage :
  \begin{itemize}
   \item \'Etat de l'art sur les méthodes d'extraction des dates de semis et d'estimation des rendements par télédétection
   \item Mise en place d'une chaîne de prétraitements des données et d'extraction des caractéristiques biophysiques des cultures
   \item Evaluation des dates de semis
   \item Estimation des rendements
   \item Rédaction du mémoire de stage.
  \end{itemize}
La planification de ces tâches et leur détail sont illustrés \`a travers le diagramme de Gantt présenté dans l'annexe \ref{annexe-a}.

\vspace{5mm} %5mm vertical space

Dans les lignes qui suivent, nous débuterons par un exposé synthétique sur les méthodes d'extraction des dates de semis et d'estimation des rendements par télédétection. 
Ceci aboutira sur la présentation des méthodes adoptées dans ce travail et nous conclurons notre étude après la présentation des résultats obtenus et leur analyse critique.