\section{Contexte du stage}

Le présent stage s'inscrit dans le cadre de recherches portant sur l’évaluation spatialisée par télédétection des pratiques d’intensification écologique des systèmes de cultures à base de mil et d’arachide au Sénégal, dans la région du bassin arachidier. Ces travaux sont relatifs aux projets GloFoodS SERENA, SIIL SIMCo et TOSCA LYSA et sont conduits par 
l'\acrshort{aida} (Cirad --- Montpellier, France) conjointement avec ses partenaires sénégalais le \acrshort{cse} et l'\acrshort{isra} (Dakar, Sénégal).

  \subsection{Présentation des organismes d'accueil}
Ce stage proposé par l’UR Aïda du Cirad s’est déroulé dans les locaux du CSE à Dakar au Sénégal.
    
    \paragraph{L'UR AÏDA}\footnote{\url{https://ur-aida.cirad.fr/}} est une unité de recherche du Cirad créée en 2014 et faisant partie du
département Persyst. Elle se positionne sur l’intensification et la durabilité de la pro-
duction des cultures annuelles en milieu tropical contraint et ses recherches visent la
pleine valorisation des ressources disponibles, en mobilisant les processus écologiques
qui régissent leur dynamique au sein des agrosystèmes. Elle a pour objectif l’étude,
la conception et la proposition de systèmes de culture annuels (canne à sucre, coton-
nier, riz \ldots) répondant aux exigences de performances agronomiques, technologiques
et environnementales. L’unité est structurée en 5 équipes de recherche dont Artists
(Télédétection, systèmes d’information, techniques de simulations et analyses spa-
tiales) pour couvrir les différents aspects de l’intensification écologique et répondre
à la demande sociétale et aux besoins du développement. Ces équipes interagissent
entre elles pour relever ensemble trois grands défis de recherche liés aux systèmes de
culture innovants : la compréhension du fonctionnement de l’agro-écosystème com-
plexe, l’émergence de systèmes de cultures innovants, efficaces et pertinents pour les
agricultures familiales des pays du Sud et l’évaluation de ces systèmes de culture
comme élément essentiel pour la décision et l’action. L’unité dispose d’une large ca-
pacité d’expertise dans les domaines des filières agricoles, de l’environnement, de la
gestion de données ou encore des systèmes d’information et propose des outils, logi-
ciels et analyses à destination des chercheurs et professionnels des filières agricoles.
    
    \paragraph{Le CSE}\footnote{\url{https://www.cse.sn/index.php/fr/}} est un organisme étatique créé en 1986. Il est placé sous la tutelle technique
du ministère en charge de l’environnement et est doté d’une personnalité morale lui
permettant de jouir d’une autonomie administrative et financière. Le CSE a pour mis-
sion de contribuer à la connaissance et à la gestion durable des ressources naturelles et de l’environnement, par la production et la diffusion de produits et services d’aide
à la décision notamment pour l’Etat, les collectivités locales, le secteur privé, la so-
ciété civile, les institutions de recherche et de développement, les organisations de
producteurs et les partenaires au développement. Les domaines d’activités du CSE
couvrent la gestion du littoral, le suivi des zones de parcours, des feux de brousse
et de la production agricole, les études de vulnérabilité et d’adaptation aux change-
ments climatiques, la séquestration de carbone, le suivi à long terme des écosystèmes
ou encore les problématiques d’environnement et santé. Le CSE dispose d’un impor-
tant réseau de partenariat au niveau national (ministères, universités, institutions de
recherches basées à Dakar à l’instar de l’IRD ou du Cirad) comme international (en
Afrique notamment le Centre Agro-hydro-météorologique (AGRHYMET) ou ailleurs
dans le monde comme la FAO ou le PNUE).
    
    \paragraph{L'ISRA}\footnote{\url{https://www.isra.sn}} est un établissement public à caractère scientifique et technologique fondé
en 1974, actuellement sous la tutelle du ministère de l’Agriculture mais disposant de
son propre conseil d’administration. L’ISRA est considéré comme la principale organisation de recherche au Sénégal employant plus de 70\% de chercheurs. L’ISRA a
pour principales missions : la conception et l’exécution de programmes de recherche
sur les productions végétales, forestières, animales et halieutiques et en économie rurale ; la création de connaissances scientifiques, l’innovation technologique et la mise au point d’outils d’aide à la décision pour l’amélioration du secteur agricole ; la valorisation et le transfert des résultats de la recherche ; la promotion et la formation à la recherche par la recherche et le développement de la coopération scientifique aussi
bien interafricaine et internationale qu’avec les institutions de recherche et universités sénégalaises.
    
  
  \subsection{Projets GloFoodS SERENA, SIIL SIMCo et TOSCA LYSA}
  
  \paragraph{SERENA}\footnote{\url{https://www.projects.igeo.fr/projet-1/}} (de la biodiverSité des paysagEs agRicoles à la sEcurité alimeNtAire des
ménages ruraux) est un projet financé dans le cadre du métaprogramme GloFoodS
(Transitions pour la sécurité alimentaire mondiale), un métaprogramme conjoint \acrshort{inra} et Cirad qui a pour objectif de mobiliser les forces scientifiques pluridisciplinaires des deux établissements pour contribuer à éclairer les mécanismes, relevant des écosystèmes et des systèmes socio-économiques, qui sous-tendent les différentes dimensions de la sécurité alimentaire. Le projet SERENA a été proposé par l’UR Aïda et
ses partenaires sénégalais (le CSE, l’ISRA, l’\acrshort{ucad} avec l’objectif d’éclairer la contribution d’un paysage agricole sénégalais diversifié à la sécurité alimentaire et nutritionnelle (SAN) des ménages ruraux. Ce projet est construit autour de l’hypothèse
que la production d’aliments au sein de paysages agricoles polyvalents et diversifiés
peut permettre d’accroître significativement la productivité des systèmes agricoles
et favoriser les sources de revenus et/ou l’accès à des produits diversifiés. Il s’agit
donc de mieux prendre en compte la dimension paysagère dans les études portant
sur la SAN. L’intérêt et l’originalité du projet SERENA est de proposer une démarche
de recherche mobilisant de façon innovante l’écologie du paysage, la télédétection,
la modélisation spatialisée et des enquêtes socio-économiques de ménage pour traiter plusieurs aspects de la SAN à l’échelle d’un paysage sénégalais. Pour ce faire, le projet s’articule autour de 4 axes : l’axe \emph{diversité paysagère} (proposer un plan d’échantillonnage spatialisé à partir d’indicateurs de diversité paysagère et cartographier plus
finement des indicateurs de la biodiversité des paysages agricoles à partir d’images
satellite), l’axe \emph{impact sur les rendements} (analyse du lien entre diversité paysagère et
rendements à l’échelle des parcelles et du paysage), l’axe \emph{SAN et moyens d’existence}
(combiner enquêtes socio-économiques sur les ménages agricoles et indicateurs de
biodiversité paysagère pour renseigner le lien entre diversité paysagère observée et
les stratégies des ménages associées en matière de sécurité alimentaire) et l’axe \emph{Valorisation} (diffusion des résultats du projet auprès de la communauté scientifique et des différents acteurs, à travers un site web intégrant une composante de webmapping et
formations prévues)
  
  
  \paragraph{SIMCo} (Sustainable Intensification of Millet based agrosystems using Cowpea in the Groundnut Basin - Senegal) est un projet financé par le SIIL\footnote{\url{http://www.k-state.edu/siil/index.html}}, laboratoire de l’Univer-
sité du Kansas. Les recherches de ce projet visent une meilleure compréhension des
mécanismes impliqués dans les associations culturales entre mil et niébé et l’estima-
tion de leurs rendements sur divers sites du bassin arachidier sénégalais. De manière
spécifique, les objectifs du projet SIMCo sont : (1) concevoir des agrosystèmes du-
rables à base de céréales et légumineuses en calibrant et en validant des modèles
de cultures associées ; et (2) estimer les rendements des cultures associées au niveau
local et régional. Pour ce faire, le projet s’articule autour de 3 grandes étapes : l’ac-
quisition des données de terrain pour la calibration des modèles de cultures, une
étude comparative entre les systèmes de cultures à base de mil et les systèmes de
cultures associées afin de mettre en lumière l’impact de l’association culturale sur les
rendements et l’évaluation spatialisée des rendements estimés à l’échelle régionale.
Le projet SIMCo est coordonné par l’ISRA et ses partenaires l’\acrshort{ird} et le Cirad.
  
  \paragraph{LYSA}\footnote{\url{https://www.projects.igeo.fr/projet-2/}} (from Landscape diversity to crop Yield monitoring in complex Smallholder
Agricultural systems : a remote sensing approach based on multi-source dense time
series of high spatial resolution imagery) est un projet financé dans le cadre de l’appel
à projet de recherches du TOSCA (CNES). Le projet LYSA se propose d’étudier le lien
entre la diversification des paysages agricoles et la sécurité alimentaire des ménages
ruraux en se focalisant notamment sur l’impact de la diversité des cultures et/ou
diversité/hétérogénéité du parc arboré sur la disponibilité alimentaire au travers du
prisme des rendements des principales céréales alimentaires d’Afrique sahélienne.
En particulier, l’objectif général du projet LYSA est de tester les potentialités offertes
par des séries temporelles multi-sources denses et à haute résolution spatiale pour :
(1) améliorer la caractérisation des paysages agricoles, en considérant notamment la
structure du parc arboré et la diversité des espèces ; (2) améliorer l’estimation et la
prévision des rendements des principales cultures céréalières alimentaires et ; (3) éva-
luer l’effet de la diversité paysagère (composition et structuration) sur les rendements.
La démarche du projet LYSA s’articule autour de 4 grandes étapes : le prétraitement
des données satellitaires, la caractérisation de la diversité paysagère (diversité des
cultures et pratiques agricoles, caractérisation du parc arboré, création d’indicateurs
synthétiques de diversité paysagère), l’estimation des rendements céréaliers à partir de séries temporelles optique et radar en se basant sur un réseau de parcelles sur
le terrain et la caractérisation de l’impact de la diversité des paysages sur les rende-
ments à l’échelle de la parcelle et du paysage. Ce projet est coordonné par l’UR Aïda
du Cirad et le CSE.

\section{Objectifs et Hypothèses}

L'objectif de ce stage est de mener une première analyse des potentialités offertes par une série temporelle multisource à haute résolution spatiale, pour un suivi spatialisé des systèmes de cultures à base de mil et d’arachide, à Diohine (près de la ville
de Niakhar, région de Fatick) dans le bassin arachidier sénégalais. De manière plus concrète, il s'agira :
  \begin{itemize}
   \item d'évaluer les dates de semis sur les différentes parcelles
   \item et d’estimer les biomasses végétatives et rendements grains du mil et gousses
de l’arachide.
  \end{itemize}
 
\vspace{5mm} %5mm vertical space

Pour ce faire, une série temporelle multisource composée d'images \emph{PlanetScope}, \emph{RapidEye} et \emph{Sentinel-2} couvrant presque entièrement 
la saison agricole 2017 sera utilisée. Nous disposons également de données de terrain décrivant, sur une quarantaine de parcelles de la zone, les pratiques agricoles adoptées ou encore les 
caractéristiques agronomiques dont les rendements observés. La principale contrainte pour ce travail que l'on peut considérer toutefois comme anodine, se trouve dans l'utilisation 
de solutions libres et open-source. Ce travail vient en amont d'une étude plus complète visant à évaluer à l’échelle d’un paysage, l’impact de la biodiversité paysagère sur la productivité des systèmes de cultures et les potentiels d’intensification des pratiques.

\vspace{5mm} %5mm vertical space

Nos hypothèses pour ce travail sont les suivantes :
  \begin{itemize}
   \item La pratique des agriculteurs est de semer le mil à sec avant les premières pluies et l’arachide dès la première pluie significative. Le début de la saison pluvieuse survient généralement vers la mi ou fin Juin, par conséquent les dates de semis estimées devraient donc se situer en Juin et les dates de fin de saison au plus
tard en fin Septembre ou début Octobre selon les dates de récolte observées.

   \item L’évaluation spatialisée des rendements des systèmes de culture mixte (ara-
chide/mil --- niébé) devrait être moins évidente que celle des systèmes de culture
pure en raison de la mixité du signal capté par les satellites.

  \end{itemize}

\section{Planning de travail}
Afin d'assurer sa bonne marche, le présent travail a été subdivisé en une série de 5 tâches principales à réaliser sur la période des 6 mois de stage :
  \begin{itemize}
   \item \'Etat de l'art sur les méthodes d'extraction des dates de semis et d'estimation des rendements par télédétection
   \item Mise en place d'une chaîne de prétraitements des données et d'extraction de variables permettant de décrire l’évolution de la végétation et les caractéristiques biophysiques des couverts 
   \item Evaluation des dates de semis
   \item Estimation des rendements
   \item Rédaction du mémoire de stage.
  \end{itemize}
La planification de ces tâches et leur détail sont illustrés \`a travers le diagramme de Gantt présenté dans l'\cref{annexe-a}.

\vspace{5mm} %5mm vertical space

Dans les lignes qui suivent, nous débuterons par un exposé synthétique sur les méthodes d'extraction des dates de semis et d'estimation des rendements par télédétection. 
Ceci aboutira sur la présentation des méthodes adoptées dans ce travail et nous conclurons notre étude après la présentation des résultats obtenus et leur analyse critique.
