\section{Introduction}

La \emph{sécurité alimentaire} désigne \og une situation garantissant à tout moment, à tous les êtres humains, la possibilité physique, sociale et économique 
de se procurer une nourriture suffisante, saine et nutritive leur permettant de satisfaire leurs besoins et préférences alimentaires pour mener une vie saine et active \fg{}
tel que défini par le Comité de la Sécurité alimentaire mondiale. On considère généralement \emph {quatre piliers} ou dimensions de la sécurité alimentaire : l'Accès (la plus mise 
en avant), la Disponibilité, la Qualité et la Stabilité. Aujourd'hui la notion d'Excès est également évoquée quand l'on sait qu'un régime alimentaire malsain peut être à la cause 
de l'obésité, du surpoids ou de maladies comme l'hypertension artérielle. Selon \citep{FAO2017}, 815 millions de personnes en 2016, soit plus d'une personne sur dix, 
seraient en situation d'insécurité alimentaire. Les facteurs de l'insécurité alimentaire sont divers. La pénurie d'eau, la dégradation des sols, le changement climatique, les épidémies, 
l'explosion démographique ou encore les situations de conflits sont autant de sources antérieures à une telle situation. \\L'un des défis majeurs de notre temps est de garantir à une 
population toujours plus grandissante (près de 10 milliards en 2050 selon les projections des Nations Unies), une alimentation suffisante pour couvrir ses besoins nutritionnels. 
Ainsi, nourrir 2 milliards de personnes de plus en 2050 nécessitera de revoir à la hausse la production alimentaire mondiale qui devra être globalement augmentée de 50\% . Face à ce défi, et surtout dans les pays en développement, le concept \og d'\emph{Intensification écologique} ou d'\emph{Agriculture écologiquement intensive} \fg{} trouve 
parfaitement son application. L'intensification écologique est un concept agronomique développé au \acrshort{cirad} dans le contexte de l'agriculture des pays du Sud, caractérisés par des 
rendements faibles. Il se veut de mettre au point des systèmes de production agricole utilisant de façon intensive, les processus biologiques et écologiques ainsi que leurs 
fonctionnalités naturelles, plutôt que d'utiliser de façon intensive les intrants (énergies fossiles, engrais chimiques, pesticides). L'utilisation intensive de ce facteur de 
production naturel et écosytémique, permettrait ainsi de maintenir des niveaux de rendements élevés, préservant les ressources naturelles et assurant de ce fait une durabilité des
écosystèmes cultivés \citep{Goulet2012}.\\ \`A Niakhar au Sénégal, dans la région du bassin arachidier, l'une des voies de l'intensification écologique repose sur les associations culturales 
entre céréales et légumineuses. Ainsi, l'on retrouve souvent des cultures de \emph{Mil} entresemées de \emph{Niébé} (Haricot) ou d'\emph{Arachide}. Plus encore, l'on note la 
présence d'espèces arborées sur les parcelles cultivées telle que le \emph{Faidherbia Albida}. Ces différentes pratiques permettraient non seulement d’accroître la fertilité des 
sols et par conséquent la productivité des systèmes de cultures mais aussi de diversifier les sources de revenus et d’alimentation dans les régions rurales où les moyens d’existence
des populations dépendent étroitement des cultures annuelles produites. Par ailleurs, elles permettraient également de limiter l’impact des fluctuations climatiques sur la 
production agricole. Une évaluation spatialisée des rendements des principales cultures alimentaires s'avère donc indispensable pour évaluer les performances agronomiques et 
environnementales des systèmes de cultures. Sur cet aspect, l'intérêt de recourir aux techniques de Télédétection n'est plus à remettre en cause. La Télédétection s'est révelée 
indispensable ces dernières décennies pour le suivi des cultures, la prévision des récoltes ou encore l'estimation de la biomasse et des rendements, sur des échelles régionales et 
globales. De nos jours, la mise au point d'instruments, combinant à la fois haute ou très haute résolution spatiale et haute fréquence temporelle à l'instar de \emph{Sentinel-2}, ouvre la
voie à des applications de l'ordre de la parcelle. Pour le suivi de l'agriculture en contexte tropical africain, ceci se traduit par l'affranchissement des contraintes liées à 
l'hétérogénéité des pratiques, aux parcellaires de petites tailles ou encore à la présence d’arbre dans les parcelles. 

\section{Contexte du stage}

Le présent stage s'inscrit dans le cadre de recherches portant sur l’évaluation spatialisée des pratiques d’intensification écologique des systèmes de cultures à base de Mil au 
Sénégal, dans la région du bassin arachidier par Télédétection. Ces travaux sont relatifs aux projets GloFoodS SERENA, SIMCo et LYSA et sont conduits par le \acrshort{cse} et 
l'\acrshort{isra} (Dakar, Sénégal) et l'\acrshort{aida} (CIRAD --- Montpellier, France).

  \subsection{Présentation des Organismes d'accueil}
Ce stage s'est déroulé dans les locaux du CSE
    \paragraph{Le CSE}
    
    \paragraph{L'ISRA}
    
    \paragraph{L'UR AÏDA}
  
  \subsection{Projets GloFoodS SERENA, SIMCo et LYSA}

\section{Objectifs et Hypothèses}

L'objectif de ce stage est de mener une première analyse des potentialités offertes par une série temporelle multisource à haute résolution spatiale, pour un suivi spatialisé des 
systèmes de cultures à base de Mil, à Niakhar dans le bassin arachidier sénégalais. De manière plus concrète, il s'agira :
  \begin{itemize}
   \item d'évaluer et cartographier les dates de semis sur les différentes parcelles
   \item et de calibrer un modèle statistique d'estimation des rendements.
  \end{itemize}
 
\vspace{5mm} %5mm vertical space

Pour ce faire, nous travaillerons avec une série temporelle multisource composée d'images \emph{PlanetScope}, \emph{RapidEye} et \emph{Sentinel-2} et couvrant presque entièrement 
la saison agricole 2017. Nous disposons également de données de terrain, décrivant sur une trentaine de parcelles de la zone, les pratiques agricoles adoptées ou encore les 
caractéristiques agronomiques dont les rendements observés. La principale contrainte pour ce travail que l'on peut considérer comme anodine malgré tout, se trouve dans l'utilisation 
de solutions libres et open-source. Ce travail vient en amont d'une étude plus complète visant à évaluer à l’échelle d’un paysage, l’impact de la biodiversité paysagère sur la productivité des 
systèmes de cultures et les potentiels d’intensification des pratiques.

\vspace{5mm} %5mm vertical space

Sur base des observations dans la zone de Niakhar, nos hypothèses pour ce travail sont les suivantes :
  \begin{itemize}
   \item \'Etant donné l'apparition des premiers germes de Mil vers le mois de Juillet pour une fin de saison au 
   plus grand tard en fin Septembre ou début Octobre, les dates de semi devraient se situer pendant le mois de Juin. 
   \item Pour le Mil, les rendements seraient améliorés par une association culturale en l'occurrence Mil --- Niébé mais aussi par la présence d'arbres dans la parcelle.
  \end{itemize}

\section{Planning de travail}
Afin d'assurer sa bonne marche, le présent travail a été subdivisé en une série de 5 tâches principales à réaliser sur la période des 6 mois de stage :
  \begin{itemize}
   \item \'Etat de l'art sur les méthodes d'extraction des dates de semis et d'estimation des rendements par Télédétection
   \item Mise en place d'une chaîne de prétraitements des données et d'extraction des caractéristiques biophysiques des cultures
   \item Evaluation des dates de semis
   \item Estimation des rendements
   \item Rédaction du mémoire de stage.
  \end{itemize}
La planification de ces tâches et leur détail sont illustrés \`a travers le diagramme de Gantt présenté dans l'annexe \ref{annexe-a}.

\vspace{5mm} %5mm vertical space

Dans les lignes qui suivent, nous débuterons par un exposé synthétique sur les méthodes d'extraction des dates de semis et d'estimation des rendements par Télédétection. 
Ceci aboutira sur la présentation des méthodes adoptées dans ce travail et nous conclurons notre étude après la présentation des résultats obtenus et leur analyse critique.