La \emph{sécurité alimentaire} désigne \og une situation garantissant à tout moment, à tous les êtres humains, la possibilité physique, sociale et économique 
de se procurer une nourriture suffisante, saine et nutritive leur permettant de satisfaire leurs besoins et préférences alimentaires pour mener une vie saine et active \fg{} 
tel que défini par le \citet{ComitedelaSecuriteAlimentaireMondiale2012}. On considère généralement \emph {quatre piliers} ou dimensions de la sécurité alimentaire : l'accès, la disponibilité 
(la plus mise en avant), la qualité et la stabilité. Aujourd'hui la notion d'Excès est également évoquée quand on sait qu'un régime alimentaire malsain peut être la cause de 
l'obésité, du surpoids ou de maladies comme l'hypertension artérielle. Selon \citep{FAO2017}, 815 millions de personnes en 2016, soit plus d'une personne sur dix dans le monde, étaient en situation 
d'insécurité alimentaire. Les facteurs de l'insécurité alimentaire sont divers. La pénurie d'eau, la dégradation des sols, le changement climatique, les épidémies, 
l'explosion démographique ou encore les situations de conflits sont autant de sources antérieures ou concomitantes à une telle situation. \\L'un des défis majeurs de notre temps est de garantir à une 
population toujours plus grandissante (près de 10 milliards en 2050 selon les projections), une alimentation suffisante pour couvrir ses besoins nutritionnels. 
Ainsi, nourrir 2 milliards de personnes de plus en 2050 nécessitera de revoir à la hausse la production alimentaire mondiale qui devra être globalement augmentée de 50\% . Face à ce défi notamment dans les pays en développement, le concept \og d'\emph{Intensification écologique} ou d'\emph{Agriculture écologiquement intensive} \fg{} trouve 
parfaitement son application. L'intensification écologique est un concept agronomique initié et promu par le \acrshort{cirad} dans le contexte de l'agriculture des pays du Sud, caractérisés par des 
rendements agricoles faibles. Ce concept se veut de mettre au point des systèmes de production agricole utilisant de façon intensive les processus biologiques et écologiques ainsi que leurs 
fonctionnalités naturelles, plutôt que d'utiliser de façon intensive les intrants (énergies fossiles, engrais chimiques, pesticides). L'utilisation intensive de ces facteurs de 
production naturel et écosytémique permettrait ainsi de maintenir des niveaux de rendements élevés, préservant les ressources naturelles et assurant de ce fait une durabilité des
écosystèmes cultivés \citep{Goulet2012}.\\ \`A Niakhar au Sénégal, dans la région du bassin arachidier, l'une des voies de l'intensification écologique repose sur les associations culturales 
entre céréales et légumineuses. Ainsi, l'on retrouve souvent des cultures de \emph{mil} entresemées de \emph{niébé} (haricot) ou d'\emph{arachide}. Plus encore, l'on note la 
présence d'espèces arborées sur les parcelles cultivées telle que le \emph{Faidherbia albida} qui est également une légumineuse. Ces différentes pratiques permettraient non seulement à moyen terme d’accroître la fertilité des 
sols et par conséquent la productivité des systèmes de cultures mais aussi de diversifier les sources de revenus et d’alimentation dans des régions rurales où les moyens d’existence
des populations dépendent étroitement des productions annuelles réalisées. Par ailleurs, elles permettraient également de limiter l’impact des fluctuations climatiques sur la 
production agricole. Une évaluation spatialisée des rendements des principales cultures alimentaires s'avère donc indispensable pour évaluer les performances agronomiques et 
environnementales de ces systèmes de cultures. Cela étant, l’intérêt de recourir aux techniques de télédétection s’impose comme une nécessité. La télédétection s’est révelée comme un outil d’une extrême utilité ces dernières décennies pour le suivi des cultures, la prévision des récoltes ou encore l'estimation de la biomasse et des rendements \citep{Kogan2013,Johnson2014,Leroux2016,Battude2016,Sibley2014a}, sur des échelles régionales et 
globales. De nos jours, la mise au point d'instruments combinant à la fois haute ou très haute résolution spatiale et haute fréquence temporelle à l'instar de \emph{Sentinel-2}, ouvre la
voie à des applications à des échelles parcellaires voire infra-parcellaires. Pour le suivi de l'agriculture dans le contexte tropical africain, ceci se traduit par l'affranchissement des contraintes liées à 
l'hétérogénéité des pratiques, aux parcelles de petites tailles ou encore à la présence d’arbre dans les parcelles.
