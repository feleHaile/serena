\documentclass[a4paper,12pt,oneside,chapterprefix=false]{scrbook}
%\usepackage[Sonny]{fncychap}
\usepackage[utf8]{inputenc}
\usepackage[T1]{fontenc}
\usepackage[top=2.5cm, bottom=2.5cm, left=2.5cm, right=2.5cm]{geometry}
\usepackage[osf,sc]{mathpazo}
\usepackage[citecolor=magenta,linkcolor=black,colorlinks=true,hypertexnames=false,pdftitle=Mémoire SERENA Gbodjo]{hyperref}
\usepackage{natbib}
\usepackage[english,french]{babel}
%\usepackage{setspace}
%\usepackage{makeidx}
%\usepackage[french]{nomencl}
\usepackage[nonumberlist,acronyms,xindy,nomain]{glossaries}
\usepackage{graphicx}
\usepackage{amsmath,amssymb,mathrsfs}
\usepackage{tikz}
\usepackage{pgfplots}\usepgfplotslibrary{dateplot}\usetikzlibrary{shapes,arrows}
\usepackage{pdfpages}
\usepackage{textcomp}
%\usepackage{shorttoc}
\usepackage{siunitx}
%\usepackage{fancyhdr}
\usepackage{multirow}
%\usepackage{ccaption}
%\usepackage[french]{varioref}
\usepackage[french]{cleveref}
\usepackage[automark,headsepline=true,autooneside=false,markcase=used]{scrlayer-scrpage} 
%\usepackage{scrpage2}
\usepackage{lettrine}
\usepackage{caption}

%\title{Evaluation du potentiel d’une série temporelle multi-source à haute résolution spatiale pour le suivi des cultures annuelles en petite agriculture familiale : 
%Cas du bassin arachidier sénégalais}
%\author{\textsc{Gbodjo} Yawogan Jean Eudes}
%\date{\today}

%\onehalfspacing
%\pagestyle{plain}
%\renewcommand{\emph}{\textbf}

\makeglossaries
\pgfplotsset{compat=1.15}

\newenvironment{abstract}%
{\cleardoublepage\thispagestyle{empty}\null\vfill\begin{center}%
\sffamily\bfseries \abstractname \end{center}}%
{\vfill\null}

\newenvironment{acknowledgements}%
{\cleardoublepage\thispagestyle{empty}\null\vfill\begin{center}%
\sffamily\bfseries Remerciements \end{center}}%
{\vfill\null}

\addto\captionsfrench{\renewcommand{\contentsname}{Sommaire}}
\addto\captionsfrench{\renewcommand{\listfigurename}{Liste des figures}}
%\addto\captionsfrench{\renewcommand{\appendixname}{Annexe}}
\newcommand{\crefpairconjunction}{ et }
\addto\captionsfrench{\renewcommand{\tablename}{Tableau}}
\addto\captionsfrench{\renewcommand{\figurename}{Figure}}
%\addto\captionsfrench{\renewcommand\bibname{Références bibliographiques}}

%\labelformat{appendix}{l'annexe~#1}
%\pagestyle{scrheadings}
\addtokomafont{pagehead}{\scriptsize}
\renewcommand*{\headfont}{\sffamily\bfseries}
\renewcommand*{\pnumfont}{\sffamily\bfseries}
\clearpairofpagestyles
%\lehead{\leftmark}%{} \hrulefill}
%\lohead{\rightmark}
\ihead{\MakeUppercase\leftmark}
\ohead{\rightmark}
\cfoot*{\pagemark}

\ohead{\ifstr{\rightmark}{\leftmark}{}{\rightmark}}

\begin{document}

\let\cleardoublepage\clearpage
\includepdf[pages=1]{Garde.pdf}
%\maketitle

\frontmatter
  
  \setcounter{tocdepth}{2}
  \selectlanguage {french}
    \begin{abstract}
    \thispagestyle{plain}
    
    \lettrine{L} 'intensification écologique des pratiques agricoles est assurément aujourd’hui l’une des voies les plus efficaces pour garantir la sécurité alimentaire de tous, tout en
    favorisant une agriculture durable préservant les écosystèmes et la biodiversité no-
    tamment dans les pays en voie de développement. Le présent travail s’inscrit dans le
    cadre de recherches portant sur l’évaluation spatialisée des pratiques d’intensification écologique des systèmes de cultures à base de mil et d’arachide, à Diohine (région de Fatick) dans la zone écogéographique du bassin arachider sénégalais. Il a pour objectif d’évaluer les dates de semis et de calibrer des modèles statistiques pour l’estimation de leur biomasses et rendements sur la saison agricole 2017, en se basant sur des métriques phénologiques dérivées de télédétection. Pour ce faire, nous avons utilisé une série temporelle multisource d’images satellitaires incluant des images PlanetScope, RapidEye et Sentinel-2 et un réseau de 47 parcelles. L’indice de végétation NDVI a été calculé pour les différentes images et la série temporelle de NDVI a été lissée et reconstruite avec un pas de temps journalier en employant deux méthodes de lissage : l’algorithme HANTS et la méthode de Whittaker. Les profils temporels de NDVI obtenus par les 2 méthodes étant difficilement dissociables sur base des critères qualitatifs de sélection définis, ils ont été maintenu pour la phase d’évaluation des dates de semis. Cette dernière a montré que la méthode de Whittaker était plus performante que HANTS compte tenu des variabilités plus faibles obtenues sur les écarts entre dates de semis observées et dates de début de croissance de la végétation extraites (SOS). Pour le reste, les dates de semis ont été estimées avec plus ou moins 5, 10 et 20 jours de décalage respectivement pour les parcelles d’arachide, de mil en culture pure et de mil en culture associée. En ce qui concerne l’estimation des biomasses et rendements, deux indices de végétation ont été comparés : le NDVI et le GDVI. Les modèles de régression linéaire établis pour l’arachide ont modérément expliqué les biomasses ($R^{2} = 0,56$ ; $RMSE = 1292$ \emph{kg/ha}) et rendements ($R^{2} = 0,49$ ; $RMSE = 404$ \emph{kg/ha}) tandis que pour le mil, les rendements ($R^{2} = 0,45$ ; $RMSE = 540$ \emph{kg/ha}) ont été mieux estimés que les biomasses ($R^{2} = 0,27$ ; $RMSE = 4254$ \emph{kg/ha}). Les variables explicatives dérivées du GDVI ont été plus performantes que celles du NDVI traditionnellement utilisé dans ce type d’approche. Nous avons expliqué ceci notamment par un changement dans les technologies d’observation de la Terre mobilisées dans le cadre de ce travail par rapport à la littérature existante.
    
    \vspace{8mm}
    
    \emph{Mots-clefs} : Intensification écologique, Evaluation spatialisée, Mil, Arachide, Série temporelle multisource (Sentinel-2, RapidEye, PlanetScope), Métriques phénologiques, Estimation des biomasses et rendements.

    \end {abstract}
    
  \selectlanguage {english}
    \begin {abstract}
    \thispagestyle{plain}
    
    \lettrine{N}owadays, the ecological intensification of agricultural practices had became undoubtedly one of the most effective way of ensuring food security for all, while promoting sustainable agriculture that preserves ecosystems and biodiversity particularly in developing countries. This work is part of researches on spatial evaluation of the ecological intensification practices of millet and groundnut-based crop systems at Diohine (Fatick region) in the Senegalese groundnut Basin. Its objective is to
    evaluate sowing dates and to calibrate statistical models for estimating biomass and
    yields on the 2017 agricultural season, based on phenological metrics. To do this, we
    used a multisource time series of satellite images including PlanetScope, RapidEye
    and Sentinel-2 images and a network of 47 field plots. The popular vegetation in-
    dex NDVI were extracted from the images and the NDVI time series was smoothed
    and reconstructed with a daily time step using two smoothing methods : HANTS
    algorithm and Whittaker smoother. The temporal profiles of NDVI obtained by the
    2 methods being difficult to dissociate on the basis of our qualitative criteria of selection, we maintained them for the sowing dates evaluation. It was later showed
    that Whittaker’s method was more efficient than HANTS, given the lower variability
    obtained on the gap between observed sowing dates and starting dates of growth of
    vegetation (SOS). For the remainder, we managed to estimate sowing dates with more
    or less 5, 10 and 20 days of interval, for the groundnut, pure millet and mixed millet
    plots respectively. With regard to the estimation of biomass and yields, we compared
    NDVI and GDVI indices. The established linear regression models moderately ex-
    plained groundnut biomass ($R^{2} = 0,56$ ; $RMSE = 1292$ \emph{kg/ha}) and yields ($R^{2} = 0,49$ ; $RMSE = 404$ \emph{kg/ha}) while millet yields ($R^{2} = 0,45$ ; $RMSE = 540$ \emph{kg/ha}) were better estimated than biomass ($R^{2} = 0,27$ ; $RMSE = 4254$ \emph{kg/ha}). The explanatory variables derived from the GDVI were more efficient than those of NDVI usually used in this type of approach. We explained this by a change in earth observation technologies mobilized as part of this work compared to the existing literature.
    
    \vspace{8mm}
    
    \emph{Keywords} : Ecological intensification, Spatial evaluation, Millet, Groundnut, Multisource time series (Sentinel-2, RapidEye, PlanetScope), Phenological metrics, Biomass and Yields estimation.
    
    \end{abstract}

  \selectlanguage {french}
  
    \begin{acknowledgements}
    \thispagestyle{plain}
    
    \vspace{5mm}
    
    {\fontfamily{lmr}\selectfont \large{Par ces simples mots, je viens tout d’abord témoigner ma gratitude à vous tous qui aviez de près ou de loin par vos soutiensmultiformes à l’aboutissement de ce travail qui vient couronner une année académique (2017-2018)
    de dur labeur et riche d’expériences.}}
    
    \vspace{5mm}
    
    {\fontfamily{lmr}\selectfont \large{Merci à toi Papa, qui n’a eu point de cesse d’\oe uvrer pour que je produise un document de qualité.}}
    
    \vspace{5mm}
    
    {\fontfamily{lmr}\selectfont \large{Ma gratitude va également à tout le corps professoral du Master SIGMA qui par leurs expertises et la qualité de leurs enseignements m’ont formé pour être parmi les meilleurs dans le domaine de la Géomatique.}}
    
    \vspace{5mm}
    
    {\fontfamily{lmr}\selectfont \large{Un infini merci à mes maîtres de stage, Dr. Louise \textsc{Leroux} et Dr. Abdoul
    Aziz \textsc{Diouf} avec qui j’ai pris plaisir à travailler tout au long de ce stage.
    Merci encore pour vos orientations et conseils quand par moment j’étais
    perdu. J’ai apprécié à leur juste valeur en ces moments précis, vos savoir-
    faire qui ont été pour moi d’une utilité inexprimable.}}
    
    \vspace{5mm}
    
    {\fontfamily{lmr}\selectfont \large{Merci également aux agents du CSE qui ont grandement contribué à mon intégration en m’accueillant comme leur frère et ami : Mme. \textsc{Sene}, Mme \textsc{Soti}, M. \textsc{N’dao}, M. \textsc{Thiaw} \emph{Jërëngen jëf}.}}
    
    \vspace{5mm}
    
    {\fontfamily{lmr}\selectfont \large{Enfin, merci spécial à mon tuteur enseignant M. David \textsc{Sheeren} qui
    a encadré ce travail et à vous messieurs les membres du jury qui allez le
    juger, recevez ici la marque de ma sincère reconnaissance.}}
    
    \end{acknowledgements}

  %\shorttoc{Sommaire}{2}
  \tableofcontents
  \listoffigures
  \listoftables
  \newacronym{cirad}{Cirad}{Centre de Coopération Internationale pour la Recherche Agronomique et le Développement} 
\newacronym{aida}{UR Aïda}{Unité de Recherche Agroécologie et intensification durable des cultures annuelles}
\newacronym{isra}{ISRA}{Institut Sénégalais de Recherche Agronomique}
\newacronym{cse}{CSE}{Centre de Suivi Ecologique}
\newacronym{evi}{EVI}{Enhanced Vegetation Index}
\newacronym{lsp}{LSP}{Land Surface Phenology}
\newacronym{savi}{SAVI}{Soil Adjusted Vegetation Index}
\newacronym{msavi}{MSAVI}{Modified Soil Adjusted Vegetation Index}
\newacronym{mvc}{MVC}{Maximum Value Composite}
\newacronym{bise}{BISE}{Best Index Slope Extraction}
\newacronym{hants}{HANTS}{Harmonic Analysis of Time Series}
\newacronym{gmes}{GMES}{Global Monitoring for Environment and Security}
\newacronym{msi}{MSI}{MultiSpectral Instrument}
\newacronym{cnes}{CNES}{Centre National d'\'Etudes Spatiales}
\newacronym{vci}{VCI}{Vegetation Condition Index}
\newacronym{siil}{SIIL}{Sustainable Intensification Innovation Lab}
\newacronym{cesbio}{CESBIO}{Centre d'\'Etudes Spatiales de la BIOsphère}
\newacronym{maccs}{MACCS}{Multi-sensor Atmospheric Correction and Cloud Screening}
\newacronym{maja}{MAJA}{MACCS-ATCOR Joint Algorithm}
\newacronym{idr}{IDR}{Iterative interpolation for Data Reconstruction}
\newacronym{ucad}{UCAD}{Université Cheikh-Anta-Diop, Dakar-Sénégal}
\newacronym{ird}{IRD}{Institut de Recherche pour le Développement}
\newacronym{inra}{INRA}{Institut National de la Recherche Agronomique}

\newglossaryentry{ndvi}
{
name = NDVI,
description = {Normalized Difference Vegetation Index --- Indice de végétation par différence normalisé}
}
\newglossaryentry{sos}
{
name = SOS, 
description = {Start of Season --- Démarrage de croissance de la végétation}
}
\newglossaryentry{eos}
{
name = EOS, 
description = {End of Season --- Fin de croissance de la végétation}
}
\newglossaryentry{pos}{
name = POS, 
description = {Peak of Season --- Maximum de croissance de la végétation}
}
\newglossaryentry{gsl}{
name = GSL, 
description = {Growing Season Length --- Durée de croissance de la végétation = EOS -- SOS}
}
\newglossaryentry{toa}{
name = TOA,
description = {Top of Atmosphere --- Radiance ou Réflectance au sommet de l'atmosphère}
}
\newglossaryentry{utm}{
name = UTM,
description = {Projection Transverse Universelle de Mercator; Zone 28N sur Niakhar}
}
\newglossaryentry{esa}{
name = ESA, 
description = {European Spatial Agency --- Agence Spatiale Européenne}
}

\newglossaryentry{envisat}{
name = ENVISAT, 
description = {ENVIronment SATellite ---  Satellite d'observation de la Terre de l'ESA lancé en 2002 
dont l'objectif est de mesurer de manière continue à différentes échelles les principaux paramètres environnementaux de la Terre 
relatifs à l'atmosphère, l'océan, les terres émergées et les glaces}
}

\newglossaryentry{theia}{
name = Theia, 
description = {Pôle de données et de services surfaces continentales --- a pour objectif d’accroître l’utilisation par la communauté scientifique et les acteurs publics de la donnée spatiale en complémentarité d’autres types de données, notamment les données in situ et aéroportées}
}

\newglossaryentry{muscate}{
name = MUSCATE, 
description = {Atelier de production MUlti Satellite, multi-CApteurs, pour des données multi-TEmporelles mise en place par le CNES et le CESBIO au sein de Theia}
}

\newglossaryentry{rmse}{
name = RMSE, 
description = {Root Mean Square Error, $RMSE = \sqrt{\frac{1}{n}\sum_{i=1}^{n} (y_{i} - x_{i})^2}$ avec $y$ la variable prédite, $x$ la variable observée et $n$ le nombre total d'échantillons}
}

\newglossaryentry{cv}{
name = CV,
description = {Coefficient de variation qui est le rapport de l'écart type à la moyenne exprimé souvent en pourcentage}
}

\newglossaryentry{Persyst}{
name = Persyst, 
description = {Performances des systèmes de production et de transformation tropicaux est un département scientifique du Cirad qui conduit des études sur les productions tropicales à l’échelle de la parcelle, de l’exploitation et de la petite entreprise de transformation}
}
\newglossaryentry{fapar}{
name = FAPAR, 
description = {Fraction of Absorbed Photosynthetically Active Radiation qui désigne la fraction de rayonnement solaire absorbée par les plantes dans le domaine spectral permettant la photosynthèse. Le FAPAR est une variable biophysique directement reliée à la productivité primaire de la végétation}
}
\newglossaryentry{lai}{
name = LAI,
description = {Leaf Area Index ou indice de surface foliaire est une grandeur sans dimension, qui exprime la surface foliaire d’un arbre, d’un peuplement, d’un écosystème ou d’un biome par unité de surface de sol. Il est déterminé par le calcul de l'intégralité des surfaces des feuilles de la plante sur la surface de sol que couvre cette plante}
}

  \printglossary[type=\acronymtype,title=Glossaire]
  

\mainmatter

  \chapter{Introduction}
  
  \section{Introduction}

La \emph{sécurité alimentaire} désigne \og une situation garantissant à tout moment, à tous les êtres humains, la possibilité physique, sociale et économique 
de se procurer une nourriture suffisante, saine et nutritive leur permettant de satisfaire leurs besoins et préférences alimentaires pour mener une vie saine et active \fg{}
tel que défini par le Comité de la Sécurité alimentaire mondiale. On considère généralement \emph {quatre piliers} ou dimensions de la sécurité alimentaire : l'Accès (la plus mise 
en avant), la Disponibilité, la Qualité et la Stabilité. Aujourd'hui la notion d'Excès est également évoquée quand l'on sait qu'un régime alimentaire malsain peut être à la cause 
de l'obésité, du surpoids ou de maladies comme l'hypertension artérielle. Selon la FAO (Rapport SOFI 2017), 815 millions de personnes en 2016, soit plus d'une personne sur dix, 
seraient en situation d'insécurité alimentaire. Les facteurs de l'insécurité alimentaire sont divers. La pénurie d'eau, la dégradation des sols, le changement climatique, les épidémies, 
l'explosion démographique ou encore les situations de conflits sont autant de sources antérieures à une telle situation. \\L'un des défis majeurs de notre temps est de garantir à une 
population toujours plus grandissante (près de 10 milliards en 2050 selon les projections des Nations Unies), une alimentation suffisante pour couvrir ses besoins nutritionnels. 
Ainsi, nourrir 2 milliards de personnes de plus en 2050 nécessitera de revoir à la hausse la production alimentaire mondiale qui devra être globalement augmentée de 50\% (Rapport SOFI 
2017). Face à ce défi, et surtout dans les pays en développement, le concept \og d'\emph{Intensification écologique} ou d'\emph{Agriculture écologiquement intensive} \fg{} trouve 
parfaitement son application. L'intensification écologique est un concept agronomique développé au \acrshort{cirad} dans le contexte de l'agriculture des pays du Sud, caractérisés par des 
rendements faibles. Il se veut de mettre au point des systèmes de production agricole utilisant de façon intensive, les processus biologiques et écologiques ainsi que leurs 
fonctionnalités naturelles, plutôt que d'utiliser de façon intensive les intrants (énergies fossiles, engrais chimiques, pesticides). L'utilisation intensive de ce facteur de 
production naturel et écosytémique, permettrait ainsi de maintenir des niveaux de rendements élevés, préservant les ressources naturelles et assurant de ce fait une durabilité des
écosystèmes cultivés \citep{Goulet2012}.\\ \`A Niakhar au Sénégal, dans la région du bassin arachidier, l'une des voies de l'intensification écologique repose sur les associations culturales 
entre céréales et légumineuses. Ainsi, l'on retrouve souvent des cultures de \emph{Mil} entresemées de \emph{Niébé} (Haricot) ou d'\emph{Arachide}. Plus encore, l'on note la 
présence d'espèces arborées sur les parcelles cultivées telle que le \emph{Faidherbia Albida}. Ces différentes pratiques permettraient non seulement d’accroître la fertilité des 
sols et par conséquent la productivité des systèmes de cultures mais aussi de diversifier les sources de revenus et d’alimentation dans les régions rurales où les moyens d’existence
des populations dépendent étroitement des cultures annuelles produites. Par ailleurs, elles permettraient également de limiter l’impact des fluctuations climatiques sur la 
production agricole. Une évaluation spatialisée des rendements des principales cultures alimentaires s'avère donc indispensable pour évaluer les performances agronomiques et 
environnementales des systèmes de cultures. Sur cet aspect, l'intérêt de recourir aux techniques de Télédétection n'est plus à remettre en cause. La Télédétection s'est révelée 
indispensable ces dernières décennies pour le suivi des cultures, la prévision des récoltes ou encore l'estimation de la biomasse et des rendements, sur des échelles régionales et 
globales. De nos jours, la mise au point d'instruments, combinant à la fois haute ou très haute résolution spatiale et haute fréquence temporelle à l'instar de \emph{Sentinel-2}, ouvre la
voie à des applications de l'ordre de la parcelle. Pour le suivi de l'agriculture en contexte tropical africain, ceci se traduit par l'affranchissement des contraintes liées à 
l'hétérogénéité des pratiques, aux parcellaires de petites tailles ou encore à la présence d’arbre dans les parcelles. 

\section{Contexte du stage}

Le présent stage s'inscrit dans le cadre de recherches portant sur l’évaluation spatialisée des pratiques d’intensification écologique des systèmes de cultures à base de Mil au 
Sénégal, dans la région du bassin arachidier par Télédétection. Ces travaux sont relatifs aux projets GloFoodS SERENA, SIMCo et LYSA et sont conduits par le \acrshort{cse} et 
l'\acrshort{isra} (Dakar, Sénégal) et l'\acrshort{aida} (CIRAD --- Montpellier, France).

  \subsection{Présentation des Organismes d'accueil}
    \paragraph{Le CSE}
    est une structure spécialisée dans le suivi environnemental et la gestion durable des ressources naturelles à partir 
    d’informations spatiales pertinentes et fiables.
    
    \paragraph{L'ISRA}
    
    L'\acrlong{isra}
    
    \paragraph{UR AÏDA --- CIRAD}
    
    L'\acrlong{aida}
  
  \subsection{Projets GloFoodS SERENA, SIMCo et LYSA}

\section{Objectifs et Hypothèses}

L'objectif de ce stage est de mener une première analyse des potentialités offertes par une série temporelle multisource à haute résolution spatiale, pour le suivi des systèmes 
de cultures à base de Mil, à Niakhar dans le bassin arachidier sénégalais. De manière plus concrète, il s'agira :
  \begin{itemize}
   \item d'évaluer et cartographier les dates de semi des différentes parcelles
   \item et de calibrer un modèle statistique d'estimation des rendements.
  \end{itemize}
  
Pour ce faire, nous travaillerons avec une série temporelle multisource composée d'images \emph{PlanetScope}, \emph{RapidEye} et \emph{Sentinel-2} et couvrant presque entièrement 
la saison agricole 2017. Nous disposons également de données de terrain, décrivant sur une trentaine de parcelles de la zone, les pratiques agricoles adoptées ou encore les 
caractéristiques agronomiques dont les rendements observés. Ce travail vient en amont d'une étude plus complète visant à évaluer à l’échelle d’un paysage, l’impact de la 
biodiversité paysagère sur la productivité des systèmes de cultures et les potentiels d’intensification des pratiques.

\vspace{5mm} %5mm vertical space

Ce travail se base sur les hypothèses suivantes :
  \begin{itemize}
   \item la première hypothèse
  \end{itemize}

\section{Planning de travail}
Afin d'assurer sa bonne marche, le présent travail a été subdivisé en une série de 5 tâches principales à réaliser sur la période des 6 mois de stage :
  \begin{itemize}
   \item \'Etat de l'art sur les méthodes d'extraction des dates de semi et d'estimation des rendements par Télédétection
   \item Mise en place d'une chaîne de prétraitements des données et d'extraction des caractéristiques biophysiques des cultures
   \item Evaluation des dates de semi
   \item Estimation des rendements
   \item Rédaction du mémoire de stage
  \end{itemize}
La planification de ces tâches et leur détail sont illustrés \`a travers le diagramme de Gantt présenté dans l'\ref{annexe-a}.

\vspace{5mm} %5mm vertical space

Dans les lignes qui suivent, nous débuterons par un exposé synthétique sur les méthodes d'extraction des dates de semi et d'estimation des rendements par Télédétection. 
Ceci aboutira sur la présentation des méthodes adoptées dans ce travail et nous conclurons notre étude après la présentation des résultats obtenus et leur analyse critique.
  
  \chapter{Contexte et Objectifs}
  
  \section{Contexte du stage}

Le présent stage s'inscrit dans le cadre de recherches portant sur l’évaluation spatialisée des pratiques d’intensification écologique des systèmes de cultures à base de Mil au 
Sénégal, dans la région du bassin arachidier par télédétection. Ces travaux sont relatifs aux projets GloFoodS SERENA, SIMCo et TOSCA LYSA et sont conduits par 
l'\acrshort{aida} (CIRAD --- Montpellier, France) conjointement avec ses partenaires sénégalais le \acrshort{cse} et l'\acrshort{isra} (Dakar, Sénégal).

  \subsection{Présentation des organismes d'accueil}
Ce stage s'est déroulé dans les locaux du CSE.
    
    \paragraph{L'UR AÏDA}
    
    \paragraph{Le CSE}
    
    \paragraph{L'ISRA}
    
  
  \subsection{Projets GloFoodS SERENA, SIMCo et LYSA}

\section{Objectifs et Hypothèses}

L'objectif de ce stage est de mener une première analyse des potentialités offertes par une série temporelle multisource à haute résolution spatiale, pour un suivi spatialisé des 
systèmes de cultures à base de Mil, à Niakhar dans le bassin arachidier sénégalais. De manière plus concrète, il s'agira :
  \begin{itemize}
   \item d'évaluer et cartographier les dates de semis sur les différentes parcelles
   \item et de calibrer un modèle statistique d'estimation des rendements.
  \end{itemize}
 
\vspace{5mm} %5mm vertical space

Pour ce faire, nous travaillerons avec une série temporelle multisource composée d'images \emph{PlanetScope}, \emph{RapidEye} et \emph{Sentinel-2} et couvrant presque entièrement 
la saison agricole 2017. Nous disposons également de données de terrain, décrivant sur une trentaine de parcelles de la zone, les pratiques agricoles adoptées ou encore les 
caractéristiques agronomiques dont les rendements observés. La principale contrainte pour ce travail que l'on peut considérer comme anodine malgré tout, se trouve dans l'utilisation 
de solutions libres et open-source. Ce travail vient en amont d'une étude plus complète visant à évaluer à l’échelle d’un paysage, l’impact de la biodiversité paysagère sur la productivité des 
systèmes de cultures et les potentiels d’intensification des pratiques.

\vspace{5mm} %5mm vertical space

Sur base des observations dans la zone de Niakhar, nos hypothèses pour ce travail sont les suivantes :
  \begin{itemize}
   \item \'Etant donné l'apparition des premiers germes de Mil vers le mois de Juillet pour une fin de saison au 
   plus tard en fin Septembre ou début Octobre, les dates de semi devraient se situer pendant le mois de Juin. 
   \item Pour le Mil, les rendements seraient améliorés par une association culturale en l'occurrence Mil --- Niébé mais aussi par la présence d'arbres dans la parcelle.
  \end{itemize}

\section{Planning de travail}
Afin d'assurer sa bonne marche, le présent travail a été subdivisé en une série de 5 tâches principales à réaliser sur la période des 6 mois de stage :
  \begin{itemize}
   \item \'Etat de l'art sur les méthodes d'extraction des dates de semis et d'estimation des rendements par télédétection
   \item Mise en place d'une chaîne de prétraitements des données et d'extraction des caractéristiques biophysiques des cultures
   \item Evaluation des dates de semis
   \item Estimation des rendements
   \item Rédaction du mémoire de stage.
  \end{itemize}
La planification de ces tâches et leur détail sont illustrés \`a travers le diagramme de Gantt présenté dans l'annexe \ref{annexe-a}.

\vspace{5mm} %5mm vertical space

Dans les lignes qui suivent, nous débuterons par un exposé synthétique sur les méthodes d'extraction des dates de semis et d'estimation des rendements par télédétection. 
Ceci aboutira sur la présentation des méthodes adoptées dans ce travail et nous conclurons notre étude après la présentation des résultats obtenus et leur analyse critique.

  \chapter{Synthèse bibliographique}
  
  \section{Extraction des dates de semis par télédétection}
  
  \subsection{Considérations générales} \label{sec-general}
  
Les périodes ou dates les plus appropriés pour semer dépendent de nombreux facteurs.  Nous pouvons considérer les espèces et leurs variétés, les températures 
selon la zone de production, l'humidité du sol, les objectifs de la production (date de récolte souhaitée) ou encore les pratiques culturales (semi direct sans labour, semi en sec avant les premières pluies \ldots{}). Cependant, l'on peut
s'accorder sur le rôle prépondérant que joue le facteur climatique et plus précisement le début de la saison des pluies \citep{Ingram2002, Barbier2009}. En effet, au Sénégal comme dans d'autres pays d'Afrique où l'agriculture est pluviale, le comportement des agriculteurs est
généralement celui de semer après les premiers épisodes pluvieux importants \citep{Bacci1999}. Néanmoins, il existe un risque d'échec des premières semences en cas d'avènement d'une sécheresse \citep{Marteau2011} qui peut emmener les agriculteurs à semer de nouveau.\\ \'Etant donné l'étroite relation entre les rendements agricoles et la durée du cycle de développement des céréales comme le Mil, 
des semis tardifs sont susceptibles de 
conduire à des rendements faibles en fin de saison \citep{Sivakumar1990}. Le suivi du démarrage de la saison agricole permet donc de fournir aux décideurs, une évaluation précoce des menaces potentielles 
à la production agricole et à la sécurité alimentaire. L'une des méthodes les plus courantes pour l'estimation des dates de semis dans les pays d'Afrique de l'Ouest repose sur une approche agrométéorologique ou agroclimatiques
qui consiste à appliquer des seuils sur les quantités de précipitations survenues pendant une période définie \citep{Marinho2014}. Cependant, cette méthode est limitée par les inconvénients qu'elle présente : la résolution spatiale des données qui est souvent grossière (de l'ordre des kilomètres) et l'estimation en elle même des quantités de précipitations qui peut être imprécise. 
Compte tenu de notre échelle de travail qui est celle de la parcelle, cette méthode n'est pas adaptée à notre étude et ne peut être considérée. \\Une autre approche par télédection cette fois-ci
consiste à dériver les dates du début de croissance de la végétation ou en général les \emph{métriques phénologiques} à partir de séries temporelles d'indices de végétation comme le \acrshort{ndvi} ou le \acrshort{evi} 
et à en déduire les dates de semis. Ces indices de végétation rendent compte entre autres de l’activité photosynthétique de la
végétation, de l’intensité de son métabolisme ainsi que de sa verdure \citep{Duarte2018}. Cette approche par télédétection présente non seulement l'avantage d'avoir une résolution 
spatiale beaucoup plus élevée en fonction de l'image 
satellitaire utilisée mais d'intégrer également la réponse spectrale de la végétation aux divers facteurs externes comme les pratiques agricoles. Bien évidemment, les contraintes intrinsèques à
l'acquisition des images satellitaires ne sont pas en reste. C'est notamment le cas des bruits induits par la perturbations atmosphériques (nébulosité). Il faudra donc avant exploitation des images, considérer 
par exemple l'application d'une méthode de lissage. Une autre précaution est à prendre vis-à-vis de l'influence des sols nus dans la réponse spectrale du couvert végétal 
notamment quand celui ci est clairsemé tel pour des cultures peu couvrantes comme le Mil. Il faudra considérer par exemple l'utilisation d'un indice de végétation tenant compte de l'effet des sols comme le \acrshort{savi} ou le 
\acrshort{msavi}. Il existe une foultitude de méthodes dans la littérature pour estimer le démarrage de la végétation en étudiant sa phénologie par observation satellitaire. 
  
  \subsection{Phénologie de la végétation}
  
La \emph{phénologie} 
est l’étude de l’apparition d’événements périodiques (annuels le plus souvent) dans le monde vivant, déterminée par les variations saisonnières du climat. Elle se traduit au niveau de 
la végétation, par l’ensemble des stades de développement intervenant dans le cycle de vie des plantes en l’occurrence le bourgeonnement, la croissance, la floraison et la 
sénescence \citep{Kimball2014}. La phénologie du Mil peut être par exemple décomposée en une phase \emph{végétative}, une phase \emph{reproductrice} et une phase de 
\emph{maturation}. Selon les variétés de Mil cultivées, la longueur du cycle de développement peut être de 90 à 100 jours pour les \emph{Souna} (variétés à cycle court --- 
petits grains, épis non aristés et peu photosensibles) ou de 130 à 150 jours pour les \emph{Sanio} (variétés à cycle long --- grains plus gros, épis aristés et variété photopériodique) \citep{Diouf2001}.
\\L'\'etude de la phénologie des plantes sur une échelle régionale à globale à partir d’observations satellitaires est désignée par \acrfull{lsp} \citep{Helman2018}. Le terme LSP 
tient compte du fait que le signal intercepté par les satellites provient d’une surface hétérogène et n’est pas représentatif de la réponse spectrale d’une seule espèce 
\citep{Kimball2014}. L'analyse de la LSP contribue entre autres à diverses applications comme l’étude des changements climatiques \citep{Begue2014} ou encore au suivi des
cultures. Les séries temporelles d’indices de végétation provenant de capteurs à basse résolution comme AVHRR, MODIS, SPOT Vegetation ou encore PROBA-V permettent d’évaluer 
la phénologie de la végétation sur une échelle régionale à globale. Plus récemment, quelques auteurs ont étudié la phénologie de la végétation à une échelle beaucoup plus fine en utilisant des données
à haute et très haute résolution comme ceux du satellite chinois HJ-1 A/B \citep{Pan2015} ou du satellite RapidEye \citep{Vrieling2017}.

  \subsection{Métriques phénologiques}

Le démarrage de la croissance de la végétation et plus généralement les métriques phénologiques peuvent être utilisés comme proxy à l'évaluation des dates de semis. Les métriques phénologiques
ou variables phénologiques désignent l'ensemble des stades phénologiques du cycle saisonnier de la végétation dérivés par observation satellitaire \citep{Helman2018}. Elles fournissent 
donc des indications sur la dynamique des écosystèmes.
Il s'agit usuellement (\Cref{metrics}) : 
\begin{itemize}
 \item du démarrage de croissance de la végétation ou \acrshort{sos},
 \item du pic ou du maximum de croissance de la végétation \acrshort{pos},
 \item de la fin de croissance de la végétation \acrshort{eos},
 \item et de la durée de croissance de la végétation \acrshort{gsl} : différence de temps entre le SOS et le EOS.
\end{itemize}
Pour ces premières métriques citées, il importe de noter que l'on considère à la fois la date (jour exact dans l'année ou nombre de jours dans le cas du GSL) et la valeur de l'indice de 
végétation utilisé par exemple le NDVI. D'autres métriques en plus peuvent être également dérivées : 
\begin{itemize}
 \item le niveau de base (le plus bas) au cours du cycle de développement de la végétation
 \item le milieu de croissance de la végétation
 \item le taux de croissance de la végétation au début du cycle
 \item le taux de décroissance de la végétation à la fin du cycle
 \item la petite intégrale saisonnière (cumul du SOS au EOS pour les valeurs au dessus de la base)
 \item la grande intégrale saisonnière (cumul du SOS au EOS en totalité).
\end{itemize}

\begin{figure}[htbp]
 \begin{center}
  \includegraphics[scale=0.45]{synthese_biblio/metrics.png} 
 \end{center}
 \caption[Illustration de quelques métriques phénologiques]{Illustration de quelques métriques phénologiques : \emph{a- SOS b- EOS c- Durée de la saison d- Valeur de Base (en unité d'indice de végétation) e- Milieu de la saison f- Maximum de l'indice de végétation et g- Amplitude de l'indice de végétation}, Source : \citet{Eklundh2017}}
 \label{metrics}
\end{figure}
%\clearpage

Outre leur utilisation comme proxy à l'évaluation des dates de semis, les métriques phénologiques peuvent être employées à d'autres fins comme l'estimation de rendements, exploitant 
la forte corrélation avec la biomasse en fin de saison ou encore la détection d'anomalies dans le cycle de la végétation, en considérant une référence ou une normale parmi un 
historique de saisons culturales.

\vspace{5mm}

Comme évoqué dans la \cref{sec-general}, plusieurs méthodes ont été mises au point pour dériver les métriques phénologiques à partir de séries temporelles d'indices de végétation.
\citet{Beck2006} et \citet{Atzberger2013} les ont classé en 2 catégories : 
\begin{itemize}
 \item un premier groupe de méthodes qui estime le timing des transitions phénologiques de manière isolée et indépendante les unes des autres
 \item et un second groupe qui modélise la série temporelle entière par une fonction mathématique pour estimer les métriques phénologiques.
\end{itemize}
Dans le premier groupe, sont classées surtout les méthodes par seuillage sur les valeurs d'indices de végétation ou sur leurs amplitudes. Dans leur revue \citet{deBeurs2010} en ont décrit quelques unes comme la méthode par seuillage basée sur les ratios de NDVI \citep{White1997}. Les méthodes par seuillage sont les plus simples et les plus communes \citep{Pan2015}. En effet, elles supposent qu’un stade phénologique a commencé quand 
la valeur de l'indice de végétation atteint un certain seuil fixé \citep{Jonsson2002}. Cependant, cette approche peut s'avérer inconsistante quand le couvert végétal n'est 
pas homogène ou quand il s'agit d'écosystèmes cultivés avec plusieurs cycles culturaux au cours de l'année. De plus, un seuil n'est pas toujours transposable du fait qu'il soit bien souvent lié à sa zone d'application. La seconde catégorie regroupe les fonctions gaussiennes, les modèles logistiques et quadratiques \citep{Zhang2003,Jonsson2004}, l'analyse en composantes principales, l'analyse harmonique ou encore la transformation en ondelettes. Ces dernières méthodes sont plus poussées et plus complexes à mettre en \oe uvre. Il existe également une autre catégorie de méthodes dite de dérivées de courbes qui définissent généralement le SOS et le EOS respectivement comme les moments où l'on observe la plus grande hausse et la plus grande chute dans les valeurs de l'indice de végétation \citep{Moulin1997,Tateishi2004}.

\vspace{5mm}

Bon nombre de logiciels et applications ont été développés pour traiter les séries temporelles d'indices de végétation et en extraire les métriques phénologiques souhaitées. 
Nous avons identifié les suivants :
\begin{itemize}
 \item TIMESAT \footnote{\url{http://web.nateko.lu.se/timesat/timesat.asp?cat=0}} \citep{Eklundh2017} 
 \item SPIRITS \footnote{\url{http://spirits.jrc.ec.europa.eu/}}
 \item PhenoSat \footnote{\url{http://www.fc.up.pt/PhenoSat/software.html}} \citep{Rodrigues2013}
 \item Plugin QGIS QPhenoMetrics \citep{Duarte2018}
 \item Plugin QGIS VERSAO VegaMonitor \footnote{\url{https://github.com/Xdarii/VERSAO_VegaMonitor/wiki}}.
\end{itemize}

\section{Estimation des rendements par télédétection}


  \chapter{Matériels et Méthodes}
  
  \section{Présentation générale de la Zone d'étude}

\section{Données satellitaires utilisées}

Notre série temporelle est composée de 3 types d’imageries satellitaires. Il s'agit d'images PlanetScope, RapidEye et Sentinel-2. 

  \subsection{Imagerie PlanetScope et RapidEye}
  
Les images PlanetScope sont produites par la société privée américaine Planet Labs, Inc. \footnote{\url{https://www.planet.com/}}. Fondée en 2010 et basée à San Francisco 
en Californie, cette société est spécialisée dans l'observation de la terre par imagerie satellitaire. Planet Labs conçoit et fabrique des \emph{nanosatellites} ou \emph{CubSats} appelés \emph{Doves}
qui sont placés sur orbite en tant que charge utile secondaire sur d'autres missions de lancement de fusée. La société dispose ainsi d'une constellation de nanosatellites surnommée \emph{Flock}. 
La constellation est composée au total de 88 nanosatellites en 2017. Ces nanosatellites produisent des images complètes de la Terre une fois par jour à une résolution 
spatiale de 3 à 5 mètres. Ces images fournissent des informations permettant de suivre les changements climatiques, de prévoir les récoltes, de gérer les catastrophes ou encore de 
mettre au point des applications urbaines. Les images recueillies par les Doves sont accessibles en ligne et certaines disponibles dans le cadre de l'Open Data. \\
En 2015, Planet Labs a acquis la constellation RapidEye auprès de la société allemande BlackBridge. Les images RapidEye qui ont une résolution spatiale de 5 mètres sont fournies par une 
constellation de 5 satellites. Leur période de revisite est de 5 jours et demi au nadir ou journalière sinon. Enfin en 2017, Google vend sa filiale Terra Bella et sa constellation de 
satellites \emph{SkySat} à Planet Labs. Les images SkySat ont une résolution submétrique (80 cm). 

    \paragraph{Spécifications des images PlanetScope}

Trois niveaux de traitements sont disponibles pour les images PlanetScope. Ce sont les niveaux 1B, 3A et 3B. Le niveau 1B correspond aux produits basiques. Les données numériques ont 
été calibrées en radiance \acrshort{toa} mais les images ne sont pas géoréférencées. Le niveau 3B qui est celui de nos images correspond à des produits orthorectifiés et projetés en
\acrshort{utm}. Comme pour le niveau 1B, les valeurs numériques ont été calibrées en radiance TOA. Le niveau 3A est similaire au 3B à la différence que les images sont tuilées pour 
couvrir un système de grilles de $25\times25$ kilomètres.\\
Les images PlanetScope sont distribuées en format \emph{GeoTiff}. Au niveau 3B, elles ont une résolution au sol de \emph{3 mètres} et une résolution radiométrique de \emph{12 bits} s'il 
s'agit de compte numérique ou \emph{16 bits} dans le cas des radiances TOA. Elles disposent de 4 bandes spectrales (\Cref{planetscope}). 

\begin{table}
\begin{center}
\caption{Caractéristiques spectrales des images PlanetScope}
\label{planetscope}
 \begin{tabular}{ccc}
  \hline
  Bande spectrale & Domaine spectral & Longueurs d'onde (micromètres)\\
  \hline
  1 & Bleu & 0,455 --- 0,515 \\
  2 & Vert & 0,500 --- 0,590 \\
  3 & Rouge & 0,590 --- 0,670 \\
  4 & Proche Infrarouge & 0,780 --- 0,860 \\ 
  \hline
 \end{tabular}
\end{center}
\end{table}

La conversion des valeurs de pixels en réflectance TOA pour les images PlanetScope est donnée par la formule suivante :
\[
   Reflectance (i) = DN(i) \times reflectanceFactor(i)
\]

  \paragraph{Spécifications des images RapidEye}

Les images RapidEye sont pour leur part, disponibles en 2 niveaux de traitements : 1B et 3A. Ces niveaux de traitements sont identiques à ceux des produits PlanetScope. Les images
RapidEye utilisées sont traitées au niveau 3A. Elles sont distribuées également en format \emph{GeoTiff}. Leur résolution au sol est de \emph{5 mètres} et leur résolution 
radiométrique de \emph{16 bits}. Elles disposent de 5 bandes spectrales (\Cref{rapideye}).

\begin{table}
\begin{center}
\caption{Caractéristiques spectrales des images RapidEye}
\label{rapideye}
 \begin{tabular}{ccc}
  \hline
  Bande spectrale & Domaine spectral & Longueurs d'onde (micromètres)\\
  \hline
  1 & Bleu & 0,440 --- 0,510 \\
  2 & Vert & 0,520 --- 0,590 \\
  3 & Rouge & 0,630 --- 0,685 \\
  4 & Red Edge & 0,690 --- 0,730 \\ 
  5 & Proche Infrarouge & 0,760 --- 0,850 \\
  \hline
 \end{tabular}
\end{center}
\end{table}

La conversion des valeurs de pixels en réflectance TOA pour les produits RapidEye se fait en 2 étapes :
\[  Radiance (i) = DN(i) \times radiometricScaleFactor(i) \]

\[   Reflectance(i) = Radiance(i) \times \pi \]

  \subsection{Imagerie Sentinel-2}

Sentinel est une famille de 6 satellites d'observation de la Terre développée par l'\acrshort{esa} et destinée à assurer la continuité des données de la mission \acrshort{envisat} arrivée à 
terme en 2012. Sentinel représente le volet spatial du programme Copernicus ou ex \acrshort{gmes} de l'Union Européenne qui vise à doter l'Europe d'une capacité autonome et 
opérationnelle en matière d'observation de la Terre notamment pour la surveillance de l'environnement et la sécurité. Sentinel-2 
 
\vspace{5mm}
 
Nous avons mis en place une chaîne de traitements écrit en \texttt{Python} découpage mosaiquage
  
\section{Données de terrain}
  
\section{Extraction d'indices spectraux}

\section{Lissage et Interpolation des Série temporelles}

Nous disposons \`a présent d'une série temporelle, pour chaque indice de végétation calculé. Cependant, avant de pouvoir les exploiter pour la suite de notre travail, un autre 
traitement s'impose : le \emph{lissage} ou \emph{smoothing}. En effet, bien que des prétraitements soient effectuées sur les images satellitaires : calibrations radiométriques et 
corrections atmosphériques entre autres, il subsiste du bruit qui affecte l'utilisation des séries temporelles d'images, impactant ainsi les futures analyses et pouvant donc fausser les interprétations
données aux résultats obtenus \citep{Chen2004}. Ce bruit résiduel peut être lié à plusieurs facteurs notamment les conditions atmosphériques variables et la présence de pixels nuageux
indétectés. Les techniques de lissage font l'hypothèse que le bruit résiduel dans les images entraine des chutes soudaines dans le profil temporel des indices de végétation 
\citep{Bojanowski2009}. Ces valeurs peuvent être ainsi identifiées puis supprimées et des séries temporelles de meilleure qualité reconstruites.
\\L'une des techniques usuelles adoptée notamment par les fournisseurs d'indices de végétation périodique comme GIMMS-MODIS, SPOT-VEGETATION ou PROBA-V est le \acrshort{mvc}. Cette technique 
consiste à créer des synthèses d'indices de végétation sur une période donnée, généralement une décade en considérant pour chaque pixel la plus grande valeur enrégistrée sur la période.
D'autres techniques de lissage de séries temporelles d'indices de végétation ont été mises au point. Nous pouvons citer des méthodes de seuillages comme l'algorithme \acrshort{bise}, 
des méthodes basées sur l'analyse de Fourier comme \acrshort{hants} \citep{Verhoef1996} et d'autres approches par fonction asymétrique notamment gaussienne \citep{Jonsson2002}. 
\citep{Chen2004} ont introduit une méthode de lissage basée sur le filtre de Savitzky-Golay \citep{Savitzky1964}. D'autres approches ont aussi été distinguées par \citep{Geng2014,
Liu2017} comme la technique de lissage de Whittaker ou encore la transformation en ondelettes.
\\Plusieurs études ont comparées ces différentes techniques de lissage entre elles \citep{Jonsson2002,Chen2004,Hird2009,Geng2014,Shao2016,Liu2017}. La conclusion faite par \citep{
Geng2014} qui ont comparé notamment 8 méthodes de lissage est qu'il n'y avait pas de meilleure méthode qu'une autre chacune présentant ses avantages et inconvénients. De plus, la
comparaison entre diverses n'est pas toujours significative du fait des différentes données utilisées et de l'application à des zones différentes. Néanmoins, s'il est vrai qu'il n'y 
a pas de méthode de lissage idéale, il n'en demeure pas moins que le filtre de Savitzky-Golay est une méthode qui revient souvent dans la littérature et qui est connue pour ses 
résultats consistants. En effet, le filtre de Savitzky-Golay est apprécié pour sa capacité à conserver la forme du profil temporel des indices de végétation ainsi que le timing et 
l'amplitude des minima et maxima locaux. Nous avons retenu et comparé 2 techniques de lissage pour nos séries temporelles : HANTS et le filtre de Savitzky-Golay.

\subsection{Lissage avec HANTS}

\subsection{Lissage avec le filtre de Savitzky-Golay}

\subsection{Interpolation}

\section{Extraction des Métriques Phénologiques}
afin d'évaluer les dates de semis
Compte tenu de notre objectif qui est celui d'atteindre un suivi spatialisé des systèmes de cultures,

\section{Estimation des Rendements}

  \chapter{Résultats et Discussions}
  
  \section{Présentation des Résultats}
  
\section{Analyse et Discussion}
  
  \chapter{Conclusion et Perspectives}
  
  Ce travail a été réalisé dans le contexte de l’évaluation spatialisée par télédétection
des systèmes de cultures à base de mil et d’arachide dans le bassin arachidier du
Sénégal. Dans cette perspective, il avait pour objectif principal d’évaluer le potentiel
d’une série temporelle multisource (Sentinel-2, RapidEye et PlanetScope) à travers
deux sous objectifs : l’évaluation des dates de semis et l’estimation des biomasses vé-
gétatives et des rendements grains du mil et gousses de l’arachide.
\\En lissant les profils temporels de NDVI extraits à partir de la série multisource,
notamment avec la méthode de Whittaker, nous avons extrait différentes métriques
phénologiques dont les dates de début de croissance de la végétation (SOS). Avec les
SOS extraits, nous avons estimé les dates de semis des différents systèmes de culture
avec notamment plus ou moins 5 jours de décalage pour les parcelles d’arachide et
10 à 20 jours de décalage pour celles de mil. Pour l’estimation des biomasses et ren-
dements, le NDVI, qui est traditionnellement utilisé, a été comparé à un autre indica-
teur de productivité de la végétation, le GDVI. Nous avons montré que le NDVI était
moins performant que le GDVI pour expliquer la variabilité de la biomasse et des
rendements, ceci s’expliquant notamment par un changement dans les technologies
d’observation de la Terre par rapport aux études précédentes. Cette étude a montré
le réel intérêt de combiner différentes sources d’images satellitaires afin d’améliorer
la répétitivité de l’information spatiale. Elle a également montré que la différence de
résolutions spatiales entre les types d’images ne représente en aucun cas un frein à
leur utilisation combinée au sein d’une même application.
\\Néanmoins, au vu des résultats obtenus, la méthodologie mise en place pourrait être
affinée, notamment pour améliorer la partie estimation des biomasses et rendements.
Dans un ordre de priorité, notre première recommandation concerne la phase de
prétraitements des images qui pourrait inclure des corrections atmosphériques afin
d’uniformiser tout le jeu données. En effet, seules les images Sentinel-2A dans notre
série temporelle multisource présentaient des corrections atmosphériques et nom-
breuses sont les études en télédétection ayant fait état du gain significatif que ces
corrections pouvaient apporter. Dans notre série temporelle, nous n’avons utilisé que
les images Sentinel-2A. Sentinel-2B étant aujourd’hui opérationnel, un couplage des
2 types d’images viendrait densifier notre série temporelle afin d’éviter des périodes
sans informations comme cela a été le cas pendant le mois de Septembre dans notre
étude. Également sur ce point, l’utilisation d’images radar pourrait être une solution
envisageable quand on sait que le principal avantage de l’imagerie radar se trouve
dans l’affranchissement des conditions atmosphériques, nuages notamment. L’adop-
tion d’images radars permettrait également de tester un apport éventuel des indices
radar pour l’extraction des SOS et EOS et l’estimation des biomasses et rendements.
En ce qui concerne les méthodes de lissage, s’il est vrai qu’elles ont plutôt donné
satisfaction, il n’en est pas moins que cette étape influence fortement les résultats
obtenus par la suite. Il serait alors intéressant de tester d’autres types d’approches
comme les fonctions asymétriques gaussiennes ou les doubles fonctions logistiques dont les résultats sont jugés tout autant satisfaisants. Aussi, la grande majorité de
notre travail a été basé sur l’utilisation du NDVI qui a montré ses limites pour l’es-
timation des biomasses et rendements. Dès lors, il serait pertinent de se tourner vers
d’autres indices de végétation comme cela a déjà pu être le cas avec le GDVI. D’autres
indicateurs comme le MSAVI qui tient compte de l’effet des sols (intéressant dans
le cas de cultures peu couvrantes comme le mil), le NDWI pouvant détecter la pré-
sence de stress hydrique chez les cultures ou encore les indices explorant la bande
du red edge comme le Red Edge NDVI pourraient être de réelles alternatives. Par
ailleurs, l’extraction des métriques phénologiques a été réalisée uniquement à partir
de la méthode de seuillage relatif sur les amplitudes de NDVI et il serait intéressant
de tester d’autres méthodes notamment les méthodes dérivatives. Outre cette étape,
celle de l’estimation des biomasses et rendements a tenu compte de variables expli-
catives provenant des seuls indices que sont le NDVI et le GDVI. Nous n’avons ni
exploré les variables biophysiques comme le LAI ni les variables climatiques comme
l’humidité du sol et les quantités de précipitations bien connues pour améliorer les
modèles calibrés. Les indices texturaux pourraient être également interessant à tester.
L’autre phase inexplorée également dans notre travail concerne la non prise en compte
du pouvoir explicatif des arbres (présents dans les parcelles) dans la productivité des
systèmes de cultures. Une des limites de notre travail aura été l’étroitresse de la taille
de notre jeu de données terrain rendant impossible la calibration de modèles multi-
variés plus robustes et leur validation. En effet, une taille d’échantillons plus grande
aurait permis l’implication de plusieurs variables dans nos modèles et aurait certaine-
ment amélioré les estimations de biomasses et rendements. D’autre part, augmenter
la taille de nos échantillons permettrait d’envisager le recours à des méthodes de ré-
gression plus poussées par fouilles de données telle que les forêts aléatoires d’arbres
décisionnels pour l’estimation des biomasses et rendements. Une autre limite de notre
travail repose sur la qualité des données de biomasses et de rendements observés. Ef-
fectivement, ces données ayant été collectées dans la cadre d’autres projets et donc
pas initialement pour l’estimation des biomasses et rendements, il nous aura été im-
possible de savoir jusqu’à quel point nous y fier. Sur ce point, un nouveau réseau de
parcelles est en cours de suivi pour la campagne agricole 2018. Ce nouveau jeu de
données sera ajouté à celui de la campagne 2017 sur laquelle nous avons travaillé et
viendra élargir la taille d’échantillons pour les estimations de biomasses et de rende-
ments de 2018. Pour finir, une évaluation spatialisée des rendements dans la région
du bassin arachidier sénégalais requiera avant tout un masque des principaux sys-
tèmes de cultures présents. Ce type de données est obtenu par classification. Dans le
cadre du projet S2-Agri\footnote{\url{http://www.esa-sen2agri.org/}} , un outil a été mis en place pour la production de masques
de cultures ainsi que des cartes d’occupation du sol spécialisées sur l’agriculture.

  
\appendix

\begin{figure}[htbp]
 \begin{center}
  \includegraphics[scale=0.7]{annexes/savgol_prs_1.png} 
 \end{center}
 %\caption{Quelques résultats du lissage de la série temporelle rectifiée}
 %\label{fig-lissage-prscor}
\end{figure}

\begin{figure}[htbp]
 \begin{center}
  \includegraphics[scale=0.7]{annexes/savgol_prs_2.png} 
 \end{center}
 %\caption{Quelques résultats du lissage de la série temporelle rectifiée}
 %\label{fig-lissage-prscor}
\end{figure}

\begin{figure}[htbp]
 \begin{center}
  \includegraphics[scale=0.7]{annexes/savgol_prs_3.png} 
 \end{center}
 %\caption{Quelques résultats du lissage de la série temporelle rectifiée}
 %\label{fig-lissage-prscor}
\end{figure}

% \begin{figure}[htbp]
%  \begin{center}
%   \includegraphics[scale=0.7]{annexes/savgol_prscor_1.png} 
%  \end{center}
%  %\caption{Quelques résultats du lissage de la série temporelle rectifiée}
%  %\label{fig-lissage-prscor}
% \end{figure}
% 
% \begin{figure}[htbp]
%  \begin{center}
%   \includegraphics[scale=0.7]{annexes/savgol_prscor_2.png} 
%  \end{center}
%  %\caption{Quelques résultats du lissage de la série temporelle rectifiée}
%  %\label{fig-lissage-prscor}
% \end{figure}
% 
% \begin{figure}[htbp]
%  \begin{center}
%   \includegraphics[scale=0.7]{annexes/savgol_prscor_3.png} 
%  \end{center}
%  %\caption{Quelques résultats du lissage de la série temporelle rectifiée}
%  %\label{fig-lissage-prscor}
% \end{figure}


\backmatter

\cleardoublepage
\refstepcounter{chapter}
\addcontentsline{toc}{chapter}{\bibname}
\bibliographystyle{apalike}
\bibliography{bibliographie/serena}
  
\end{document}          
