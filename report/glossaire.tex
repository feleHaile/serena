\newacronym{cirad}{Cirad}{Centre de Coopération Internationale pour la Recherche Agronomique et le Développement} 
\newacronym{aida}{UR Aïda}{Unité de Recherche Agroécologie et intensification durable des cultures annuelles}
\newacronym{isra}{ISRA}{Institut Sénégalais de Recherche Agronomique}
\newacronym{cse}{CSE}{Centre de Suivi Ecologique}
\newacronym{evi}{EVI}{Enhanced Vegetation Index}
\newacronym{lsp}{LSP}{Land Surface Phenology}
\newacronym{savi}{SAVI}{Soil Adjusted Vegetation Index}
\newacronym{msavi}{MSAVI}{Modified Soil Adjusted Vegetation Index}
\newacronym{mvc}{MVC}{Maximum Value Composite}
\newacronym{bise}{BISE}{Best Index Slope Extraction}
\newacronym{hants}{HANTS}{Harmonic Analysis of Time Series}
\newacronym{gmes}{GMES}{Global Monitoring for Environment and Security}
\newacronym{msi}{MSI}{MultiSpectral Instrument}
\newacronym{cnes}{CNES}{Centre National d'\'Etudes Spatiales}
\newacronym{vci}{VCI}{Vegetation Condition Index}
\newacronym{siil}{SIIL}{Sustainable Intensification Innovation Lab}
\newacronym{cesbio}{CESBIO}{Centre d'\'Etudes Spatiales de la BIOsphère}
\newacronym{maccs}{MACCS}{Multi-sensor Atmospheric Correction and Cloud Screening}
\newacronym{maja}{MAJA}{MACCS-ATCOR Joint Algorithm}
\newacronym{idr}{IDR}{Iterative interpolation for Data Reconstruction}
\newacronym{ucad}{UCAD}{Université Cheikh-Anta-Diop, Dakar-Sénégal}
\newacronym{ird}{IRD}{Institut de Recherche pour le Développement}
\newacronym{inra}{INRA}{Institut National de la Recherche Agronomique}

\newglossaryentry{ndvi}
{
name = NDVI,
description = {Normalized Difference Vegetation Index --- Indice de végétation par différence normalisé}
}
\newglossaryentry{sos}
{
name = SOS, 
description = {Start of Season --- Démarrage de croissance de la végétation}
}
\newglossaryentry{eos}
{
name = EOS, 
description = {End of Season --- Fin de croissance de la végétation}
}
\newglossaryentry{pos}{
name = POS, 
description = {Peak of Season --- Maximum de croissance de la végétation}
}
\newglossaryentry{gsl}{
name = GSL, 
description = {Growing Season Length --- Durée de croissance de la végétation = EOS -- SOS}
}
\newglossaryentry{toa}{
name = TOA,
description = {Top of Atmosphere --- Radiance ou Réflectance au sommet de l'atmosphère}
}
\newglossaryentry{utm}{
name = UTM,
description = {Projection Transverse Universelle de Mercator; Zone 28N sur Niakhar}
}
\newglossaryentry{esa}{
name = ESA, 
description = {European Spatial Agency --- Agence Spatiale Européenne}
}

\newglossaryentry{envisat}{
name = ENVISAT, 
description = {ENVIronment SATellite ---  Satellite d'observation de la Terre de l'ESA lancé en 2002 
dont l'objectif est de mesurer de manière continue à différentes échelles les principaux paramètres environnementaux de la Terre 
relatifs à l'atmosphère, l'océan, les terres émergées et les glaces}
}

\newglossaryentry{theia}{
name = Theia, 
description = {Pôle de données et de services surfaces continentales --- a pour objectif d’accroître l’utilisation par la communauté scientifique et les acteurs publics de la donnée spatiale en complémentarité d’autres types de données, notamment les données in situ et aéroportées}
}

\newglossaryentry{muscate}{
name = MUSCATE, 
description = {Atelier de production MUlti Satellite, multi-CApteurs, pour des données multi-TEmporelles mise en place par le CNES et le CESBIO au sein de Theia}
}

\newglossaryentry{rmse}{
name = RMSE, 
description = {Root Mean Square Error, $RMSE = \sqrt{\frac{1}{n}\sum_{i=1}^{n} (y_{i} - x_{i})^2}$ avec $y$ la variable prédite, $x$ la variable observée et $n$ le nombre total d'échantillons}
}

\newglossaryentry{cv}{
name = CV,
description = {Coefficient de variation qui est le rapport de l'écart type à la moyenne exprimé souvent en pourcentage}
}

\newglossaryentry{Persyst}{
name = Persyst, 
description = {Performances des systèmes de production et de transformation tropicaux est un département scientifique du Cirad qui conduit des études sur les productions tropicales à l’échelle de la parcelle, de l’exploitation et de la petite entreprise de transformation}
}
\newglossaryentry{fapar}{
name = FAPAR, 
description = {Fraction of Absorbed Photosynthetically Active Radiation qui désigne la fraction de rayonnement solaire absorbée par les plantes dans le domaine spectral permettant la photosynthèse. Le FAPAR est une variable biophysique directement reliée à la productivité primaire de la végétation}
}
\newglossaryentry{lai}{
name = LAI,
description = {Leaf Area Index ou indice de surface foliaire est une grandeur sans dimension, qui exprime la surface foliaire d’un arbre, d’un peuplement, d’un écosystème ou d’un biome par unité de surface de sol. Il est déterminé par le calcul de l'intégralité des surfaces des feuilles de la plante sur la surface de sol que couvre cette plante}
}
